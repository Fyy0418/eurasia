\documentclass{manual}

\title{Eurasia 用户指引}

\author{沈崴}

\authoraddress{
	项目: \url{http://eurasia.pyforce.com}
	邮件列表: \url{https://groups.google.com/group/eurasia-users}
}

\date{2009 年 3 月 23 日}

\release{3.0.0a6}

\makeindex

\begin{document}

\maketitle

\begin{abstract}

\noindent

eurasia 高性能应用服务器, 自带 shelve2 对象数据库、编译型模板、fastcgi 支持及 wsgi 服务器。
可以用于面向高并发的长连接 http 服务器, 和 tcp 服务器的设计。

\end{abstract}

\tableofcontents

\chapter{安装}

\section{系统要求}

eurasia 基于 python2.5。一般来说, 支持 python2.5 的计算机平台都能够运行 eurasia。

推荐配置:

\begin{itemize}

\item \strong{32 位 x86 体系结构}

目前尚没有证据证明 eurasia 能通过 64 位或其他体系结构来获得额外的性能提升。

\item \strong{unix/linux 操作系统}

eurasia 基于 unix 异步 IO 设计。

\item \strong{Stackless Python 2.5.2}

eurasia 基于 stackless python 协程技术 (轻便线程、微线程) 设计。所以 stackless python 是最适合的。

\end{itemize}

\section{兼容性话题}

\strong{windows 操作系统}

需要注意的是, eurasia 在 windows 上不能达到 eurasia 在 unix/linux 上的并发性能。

\strong{python 2.5}

普通的 python 2.5 可以通过 greenlet 技术来实现 eurasia 所需的 stackless 特性。在性能上并不会有明显差别。
如果你正在使用非 stackless 的 python 版本, eurasia 会自动从网上下载并安装 py 包, 你需要准备好网络连接。
因为索引会比较耗时, 这里推荐先对 py 包进行手工安装 (stackless python 不需要), 再安装 eurasia。

\strong{python 2.6}

Python 2.6 是否能够很好地兼容 eurasia, 未经充分测试。

\strong{python 3}

eurasia 目前不支持 python 3。

\section{开始安装}

Eurasia 的官方网站是 \url{http://eurasia.pyforce.com}, 可以在这里下载到 eurasia 的最新版本。

你可以通过解压目录中的安装脚本 \file{setup.py} 来安装 eurasia 库:

\begin{verbatim}
    python setup.py install
\end{verbatim}

对于非 stackless 的 python 2.5, eurasia 需要 py (\url{http://codespeak.net/py}) 包的支持。
如果没有检查到 py 包, 安装程序会尝试通过 setuptools 自动下载并安装 py 包 (推荐手工安装)。

\chapter{教程}

\section{快速开始}

我们将从最简单的 hello world 开始, 通过一些范例程序, 快速进入 eurasia 的世界。

\subsection{hello world!}

创建一个名为 \file{hello.py} 的文件。

\begin{verbatim}
#!/usr/bin/python2.5
from eurasia import config, mainloop
def handler(httpfile):
	httpfile['Content-Type'] = 'text/html'
	httpfile.write('<html>hello world!</html>')
	httpfile.close()

config(handler=handler, port=8080)
mainloop()
\end{verbatim}

执行该脚本, 启动服务器。

\begin{verbatim}
$/usr/bin/python hello.py
\end{verbatim}

这个程序简单地把处理函数 handler 绑定到本机的 8080 端口,
所有发往本机 8080 端口的 HTTP 请求都会得到一个 "hello world!" 的响应。

这样, 我们的第一个 eurasia 程序就完成了。而且这的确已经包含了 eurasia web 框架的大部分接口。

\subsection{httpfile 对象}

在上面的例子中, 我们已经接触到 httpfile 对象。
httpfile 对象封装了我们进行 web 开发要用到的所有功能。

httpfile 对象包含以下读相关的属性。

\begin{tableii}{c|l}{}{ 属性 }{ 信息 }
\lineii{ httpfile.uri            }{ 完整的请求地址, 比如 "/show?page=10" }
\lineii{ httpfile.method         }{ 请求方法, "GET" 或 "POST" }
\lineii{ httpfile.version        }{ "HTTP/1.0" 或 "HTTP/1.1" }
\lineii{ httpfile[key]           }{ 即 __getitem__ 方法。用于获取 http 头, 比如 httpfile["User-Agent"] }
\lineii{ httpfile.environ        }{ 类似于 cgi 的环境变量 }
\lineii{ httpfile.script_name    }{ uri 的路径部分, 如果 uri 是 "/show?page=10" 那么 script_name 为 "/show" }
\lineii{ httpfile.query_string   }{ uri 在请求部分, 如果 uri 是 "/show?page=10" 那么 query_string 为 "page=10" }
\lineii{ httpfile.address        }{ 客户端的 ip 与端口, 诸如 ('192.168.0.101', 4433) }
\lineii{ httpfile.read(size)     }{ 对 "POST" 请求读取指定大小的请求报文, 如果不指定 size 则全部读取 }
\lineii{ httpfile.readline(size) }{ 读取一行或指定大小的报文, 如果不指定 size 则到完成读取一行为止 (直至文件结尾) }
\end{tableii}

httpfile 对象包含以下写相关的属性。

\begin{tableii}{c|l}{}{ 属性 }{ 信息 }
\lineii{ httpfile[key]        }{ 即 __setitem__ 方法。用于设置 http 头, 比如 httpfile['Content-Type']='text/html'}
\lineii{ httpfile.status      }{ 设置响应状态, 正常响应是 200 (200 是默认值, 常见的还有 404 "未找到" 等) }
\lineii{ httpfile.write(data) }{ 发送文本 data 到客户端 }
\lineii{ httpfile.close()     }{ 完成本次请求 }
\end{tableii}

\note{httpfile 对象具有 __getitem__ 和 __setitem__ 方法, 可以像字典那样读写。
但是读取的是请求的头部, 写的是响应的头部, 所以写进去和读出来的值并不一致, 这是需要注意的。}

\subsection{一个综合应用}

下面运用 httpfile 对象编写一个简单的文件服务器。

\note{本范例仅用于演示。}

\begin{verbatim}
#!/usr/bin/python2.5
#-*- coding: utf-8 -*-

import os.path
from eurasia import config, mainloop

def handler(httpfile):
	filename = '/www' + httpfile.script_name
	if not os.uri.exists(filename): # 文件不存在  , 返回 404 (未找到)
		httpfile.status = 404
		httpfile.write('<h1>Not Found</h1>')
		return httpfile.close()

	if os.path.isdir(filename):     # 请求的是目录, 返回 403 (拒绝访问)
		httpfile.status = 403
		httpfile.write('<h1>Forbidden</h1>')
		return httpfile.close()

	try:
		data = open(filename).read()
	except (IOError, OSError):      # 文件读取错误, 返回 500
		httpfile.status = 500
		httpfile.write('<h1>Internal Server Error</h1>')
		return httpfile.close()

	# 正确返回文件内容, 因为设定了 Content-Type 为 application/octet-stream,
	# 浏览器将提示用户是否下载该文件, 而不是直接显示出来
	httpfile['Content-Type'] = 'application/octet-stream'
	httpfile.write(data)
	httpfile.close()

config(handler=handler, port=8080)
mainloop()
\end{verbatim}

\subsection{服务器配置}

我们不仅可以启动一个服务器, 还可以配置出多个服务器。

\begin{verbatim}
config(handler=handler1, port=8080)
config(handler=handler2, port=9080)
mainloop()
\end{verbatim}

除了把服务器配置到端口上, 我们还可以详细指定绑定的端口。
而且服务器不仅可以绑定到一个 ip 和端口上, 还可以绑定到多个。

\begin{verbatim}
config(handler=handler, bind='192.168.0.1:8080, 192.168.0.2:80')
\end{verbatim}

如果既不指定 "port" 也不指定 "bind", 那么服务器将会以 fastcgi 模式启动。

\begin{verbatim}
#!/usr/bin/python2.5
#-*- coding: utf-8 -*-

# 这是一个 fastcgi 程序
# 带有 mod_fcgid 或 mod_fastcgi 的 apache 可以直接启动该程序

from eurasia import config, mainloop
def handler(httpfile):
	httpfile.write('<html>hello world!</html>')
	httpfile.close()

config(handler=handler)
mainloop()
\end{verbatim}

\subsection{其他配置}

我们还可以指定调试模式、用户名等参数。

\begin{verbatim}
import eurasia
eurasia.verbose =  False   # 禁止输出调试信息, 默认为 True
eurasia.user    = 'nobody' # 以 'nobody' 的身份执行程序
\end{verbatim}


\section{表单、文件上传及 session 处理}

你可以在 eurasia 里方便地处理表单和文件上传。

表单及文件上传处理功能是由 eurasia.cgietc 模块提供的。

\begin{verbatim}
from eurasia.cgietc import Form, SimpleUpload
\end{verbatim}

另外从 eurasia.web 和 eurasia.fcgi 导入也是一样的。

\begin{verbatim}
from eurasia.web  import Form, SimpleUpload
from eurasia.fcgi import Form, SimpleUpload
\end{verbatim}

\subsection{form}

Form 函数将用户提交的表单解析成一个 dict 对象。

\begin{verbatim}
formdict = Form(httpfile)
\end{verbatim}

你可以通过设定 max_size 来限制用户 POST 数据的大小, 如果用户提交的数据超过限制, Form 函数会抛出 IOError。

\begin{verbatim}
#!/usr/bin/python2.5
import os.path
from eurasia.cgietc import Form
from eurasia import config, mainloop

def handler(httpfile):
	try:
		form = Form(httpfile, max_size=1024)
	except IOError:
		httpfile.status = 500
		httpfile.close()

	httpfile['Content-Type'] = 'text/plain'
	print >> httpfile, `form`
	httpfile.close()

config(handler=handler, port=8080)
mainloop()
\end{verbatim}

\note{如果用户提交了多个重名的表单变量, Form 会使用 list 来保存。比如 "?a=1\&a=2\&a=3\&b=4", form['a'] 的取值是 ['1', '2', '3'], 而 form['b'] 的取值是 '4'。}

\subsection{文件上传}

这是一个典型的文件提交页面。

\begin{verbatim}
<html>
<form action="/simpleupload" method="post" enctype="multipart/form-data">
<input type="hidden" name="a" value="表单变量A" />
<input type="file" name="z-file" />
<input type="submit" />
</form>
</html>
\end{verbatim}

eurasia 的 SimpleUpload 工具对文件上传提供了支持。

\begin{verbatim}
fd = SimpleUpload(httpfile)
\end{verbatim}

SimpleUpload 会返回一个上传文件的文件描述符 (fd), 我们可以从 fd.filename 和 fd.size 中取出文件名和文件大小。
其余的表单变量可以通过 fd[key] 取出。

\begin{tableii}{c|l}{}{ 属性 }{ 信息 }
\lineii{ SimpleUpload.size           }{ 上传文件的大小 }
\lineii{ SimpleUpload.filename       }{ 上传文件的文件名 }
\lineii{ SimpleUpload[key]           }{ 除了上传文件之外, 其余表单变量 }
\lineii{ SimpleUpload.read(size)     }{ 从上传文件中读取指定大小的内容, 默认读取全部内容 }
\lineii{ SimpleUpload.readline(size) }{ 行读取 }
\end{tableii}

下面是完整的处理脚本。

\begin{verbatim}
#!/usr/bin/python2.5
#-*- coding: utf-8 -*-
from eurasia import config, mainloop
from eurasia.cgietc import SimpleUpload

page = '''\
<html>
<form action="/simpleupload" method="post" enctype="multipart/form-data">
<input type="hidden" name="a" value="表单变量A" />
<input type="file" name="z-file" />
<input type="submit" />
</form>
</html>
'''

def handler(httpfile):
	if httpfile.script_name != '/simpleupload':
		httpfile.write(page) # 输出前面定义的上传表单
		return httpfile.close()

	fd = SimpleUpload(httpfile)
	a = fd['a'] # <input type="hidden" name="a" value="..."/>

	filename = fd.filename
	size     = fd.size
	data     = fd.read()  # 读取所有内容

	httpfile['Content-Type'] = 'text/plain; charset=utf-8'

	print >> httpfile, ''
	print >> httpfile, '变量 a:', a
	print >> httpfile, '文件名:', filename
	print >> httpfile, '文件预测大小:', size
	print >> httpfile, '文件实际大小:', len(data)

	httpfile.close()

config(handler=handler, port=8080)
mainloop()
\end{verbatim}

\note{SimpleUpload 要求表单变量名 (这里是 'a') 必须小于文件控件的名称 (这里是 'z-file')。
因为对 SimpleUpload 组件来说, 文件组件的名字并不重要, 所以请尽量起成 "zzzzzzzz" 之类的名字,
以保证他的值是最大的。另外, SimpleUpload 组件只支持上传一个文件。
当需要提交多个文件时, 你可以使用多个 SimpleUpload 组件。}

\note{SimpleUpload 并不会真的把用户上传的文件建立在磁盘上, 也不会把用户的上传读取到内存中,
而是在调用 read() 时才会从 socket 中去获取。因此, 在没有读取完整个上传文件之前,
上传用户的浏览器将一直保持在 "正在上传" 的状态下。}

\note{http 协议并不是每次都确切地知道用户上传文件的大小,
SimpleUpload 能够猜测出用户上传内容的大小, 对于标准的浏览器和客户端这通常是准确的。
在极少数情况下 SimpleUpload.size 和 SimpleUpload.read 取出的内容大小并不匹配,
这时应以 SimpleUpload.read 取出的实际内容大小为准, 一般来说, 你可以认为这是一种恶意行为。}

\note{在向用户发送任何内容之前, 请先确保已经完全读取了上传文件, 否则会发生读写错误 (IOError)。}

\note{在读取文件时如果用户已经断开连接, read 会抛出 IOError。}

\subsection{手工获取 post 报文}

你可以通过 Form 和 SimpleUpload 读取用户通过 post 方法提交的表单和文件。

如果用户通过 post 方法提交了一个 xml 报文或者 json 等类型的报文,
那么你就需要手工读取整个 post 报文进行处理了。httpfile.read()/httpfile.readline() 即是这样的工具。

\begin{verbatim}
#!/usr/bin/python2.5
#-*- coding: utf-8 -*-
import os.path
from eurasia import config, mainloop

def echo(httpfile):
	# 原样返回用户提交的 post 报文
	if httpfile.method == 'POST':
		httpfile['Content-Type'] = 'text/plain'
		httpfile.write(httpfile.read())
		return httpfile.close()

	# 不支持 GET 方法
	httpfile.status = 405 # Method Not Allowed
	httpfile.close()

config(handler=echo, port=8080)
mainloop()
\end{verbatim}

\note{在向用户发送任何内容之前, 请先确保已经完全读取了 post 报文, 否则会发生读写错误 (IOError)。}

\note{Form 和 SimpleUpload 即是基于 httpfile.read 实现的,
Form、SimpleUpload 和 httpfile.read 会同时读取 post 报文, 他们是不能混用的。}

\subsection{cookie \& session}

你可以通过 httpfile['Set-Cookie'] 和 httpfile['Cookie'] 这两个接口直接操作 cookie, 去实现一个自己喜欢的 session 机制。

此外, eurasia 也提供了 session 支持。通过设置 httpfile.uid, 你可以为用户的浏览器指定一个持久 id (基于 cookie), 并通过 httpfile.uid 取出。

\begin{tableii}{c|l}{}{ 属性 }{ 信息 }
\lineii{ httpfile.uid      }{ 取得持久 id }
\lineii{ httpfile.uid = id }{ 设置持久 id }
\end{tableii}

这里是一个简单的演示程序。创建并获取一个 uid。

\begin{verbatim}
#!/usr/bin/python2.5
#-*- coding: utf-8 -*-

import random, sha
from eurasia.web import config, mainloop

def handler(httpfile):
	# 如果没有 uid 就为该浏览器用户分配一个
	if not httpfile.uid:
		uid = sha.new('%s' %random.random()).hexdigest()
		httpfile.uid = uid
		print >> httpfile, 'uid %r created' %uid
		return httpfile.close()

	# 有 uid 就在页面上输出
	print >> httpfile, 'uid is %r' %httpfile.uid
	httpfile.close()

config(handler=handler, port=8080)
mainloop()
\end{verbatim}

基于持久 id 可以做出各种 session 实现。这里是一个内存 session,
我们将用户提交的表单保存在字典中, 在下一次访问时取出。
使用 dbm、bsddb、memcached 以及 MySQL 的原理相同。

\begin{verbatim}
#!/usr/bin/python2.5
#-*- coding: utf-8 -*-

import random, sha
from eurasia.cgietc import Form
from eurasia import config, mainloop

session = {}
newid = lambda: sha.new('%s' %random.random()).hexdigest()

def handler(httpfile):
	if not httpfile.uid:          # 新建 session
		uid = newid()
		while uid in session:
			uid = newid()

		session[uid] = None
		httpfile.uid = uid
	else:
		uid = request.uid

	data = session[uid]          # 前一次的表单数据
	session[uid] = Form(request) # 在 session 中保存本次的表单数据

	print >> httpfile, data      # 返回前一次的表单数据
	httpfile.close()

config(handler=handler, port=8080)
mainloop()
\end{verbatim}


\section{长连接}

首先只要请求头部指定了 "Connection: keep-alive", eurasia 在请求结束后不会立即断开连接,
而是始终和客户端保持连接。这对开发者是透明的, 你不必关心这一点或者手工维持连接。

除此之外 eurasia 还支持发送长连接响应。

\subsection{httpfile 长连接响应模式}

在通常情况下, httpfile.write() 的内容会在调用 httpfile.close() 时一次性地发送给浏览器客户端。
通过 httpfile.begin() 调用, 我们可以使 httpfile.write() 的内容即时发送给客户端。

在下面这个例子中, 每当有新用户访问站点, 所有在线用户都将收到一条提醒。

\note{需要注意的是因为 firefox 浏览器的缓存策略, 无法同时打开两个 html 长连接页面
(但是下面将要提到的 javascript 远程调用不受影响), 这不是 eurasia 引起的。
如果你正在使用 firefox , 除了 firefox 之外你还要多打开一个浏览器才能看到效果。
或者使用 telnet 发送 "GET / HTTP/1.0" 来进行测试}

\begin{verbatim}
#!/usr/bin/python2.5
#-*- coding: utf-8 -*-

from time import strftime
from eurasia import config, mainloop

# 保存了所有在线用户的全局列表
global_httpfiles = set()

def handler(current_httpfile):
	# 响应头部必须在调用 httpfile.begin() 之前设置好
	current_httpfile['Content-Type'] = 'text/html; charset=utf-8'

	# 进入长连接响应模式
	current_httpfile.begin()
	current_httpfile.write(strftime('[%a, %d-%b-%Y %H:%M:%S GMT] 我加入啦!<br/>'))

	# 通知其他在线用户, 有新人加入
	disconnected = []
	for httpfile in global_httpfiles:
		try:
			# 告诉其他在线用户有新人加入
			httpfile.write(strftime('[%a, %d-%b-%Y %H:%M:%S GMT] 又有新人加入啦!<br/>'))
		except IOError:
			# 连接已断开
			disconnected.append(httpfile)

	# 移除已经断开连接的 httpfile
	for httpfile in disconnected:
		global_httpfiles.remove(httpfile)

	# 将当前浏览器添加到全局在线用户列表中
	global_httpfiles.add(current_httpfile)

config(handler=handler, port=8080)
mainloop()
\end{verbatim}

一旦调用 httpfile.begin() 以后, httpfile.write() 会直接输出到浏览器,
httpfile.write() 函数会一直处于 "阻塞" 状态, 直至所有内容发送完毕。

如果在 httpfile.begin() 和 httpfile.write() 调用时与客户端的连接已经断开,
你会得到一个 IOError 异常。

同样使用 httpfile.close() 结束本次请求。

\note{响应头部和 httpfile.status 必须在调用 httpfile.begin() 之前设置好。}

\subsection{javascript rpc}

eurasia.cgietc 提供的 Browser 对象是一种更为强大的长连接工具。
服务器端可以在任何时候即时调用客户端的 javascript 函数 (javascript 远程调用)。

在下面这个例子中, 每当有新用户访问站点,所有在线用户都将收到一条 javascript 的提醒。

我们首先定义一个包含有 javascript 的 html 页面, 并与服务器建立一条长连接。
其中的 "my_alert" 函数就是我们即将要用到的。

\begin{verbatim}
<html>
<head>
<script language="JavaScript">
function my_alert(stuff) { alert(stuff); };
</script>
</head>
<body>
<!-- 与 /remotecall 位置的服务器脚本建立长连接 -->
<iframe src="/remotecall" style="display: none;"></iframe>
</body
</html>
\end{verbatim}

下面是完整的处理脚本。

\begin{verbatim}
#!/usr/bin/python2.5
#-*- coding: utf-8 -*-

from eurasia.cgietc import Browser
from eurasia import config, mainloop

page = '''\
<html>
<head>
<script language="JavaScript">
function my_alert(stuff) { alert(stuff); };
</script>
</head>
<body>
<!-- 与 /remotecall 位置的服务器脚本建立长连接 -->
<iframe src="/remotecall" style="display: none;"></iframe>
</body
</html>
'''

# 保存了所有在线用户的全局列表
global_browsers = set()

def handler(httpfile):
	if httpfile.script_name != '/remotecall':
		httpfile.write(page) # 输出前面定义的上传表单
		return httpfile.close()

	# 创建 Browser 对象
	# 调用客户端名为 "my_alert()" 的 js 函数
	current_browser = Browser(httpfile)
	current_browser.my_alert(u'我加入啦!')

	# 通知其他在线用户, 有新人加入
	disconnected = []
	for browser in global_browsers:
		try:
			# 调用在线用户的 my_alert 函数
			browser.my_alert(u'又有新人加入啦!')

		except IOError: # 已经断开连接
			disconnected.append(browser)

	# 移除已经断开连接的浏览器
	for browser in disconnected:
		disconnected.remove(browser)

	# 将当前浏览器添加到全局浏览器对象列表中
	global_browsers.add(current_browser)

config(handler=handler, port=8080)
mainloop()
\end{verbatim}

我们可以在 Brwoser 对象创建时指定 javascript 域 (document.domain),
方便 javascript 的跨子域调用。

\begin{verbatim}
browser = Browser(httpfile, domain=DOMAIN)
\end{verbatim}

\note{如果不指定 domain, Browser 对象会指定 domain 为顶级域名,
如果当前域为 "www.example.com" 那么默认即为 "example.com"}

\section{WSGI}

eurasia 带有 wsgi 服务器, eurasia.WSGIServer 与 flup.WSGIServer 的用法差不多。

\begin{verbatim}
#!/usr/bin/python2.5
from eurasia import WSGIServer
def app(environ, start_response):
	start_response('200 OK', [('Content-Type', 'text/html')])
	return ['<html>hello world!</html>']

server = WSGIServer(app, bindAddress=('0.0.0.0', 8080))
server.run()
\end{verbatim}

\note{除了 bindAddress, 还可以使用 port 和  bind 来指定地址, 如果不指定地址, wsgi 服务器将启动为 fcgi}

你也可以使用 config 函数来配置 wsgi 服务器。

\begin{verbatim}
#!/usr/bin/python2.5
from eurasia import config, mainloop
def app(environ, start_response):
	start_response('200 OK', [('Content-Type', 'text/html')])
	return ['<html>hello world!</html>']

config(app=app, port=8080)
mainloop()
\end{verbatim}

\note{在 wsgi application 中, 我们仍然可以从 environ['eurasia.httpfile'] 中得到 httpfile 对象}

\subsection{启动 django}

首先我们通过 django-admin.py startproject SITE 命令创建一个站点。

我们可以编写如下服务器启动脚本。

\begin{verbatim}
#!/usr/bin/python2.5
#-*- coding: utf-8 -*-
# django 服务器启动脚本

import sys, os
sys.path.insert(0, '/PATH/TO/DJANGO/PROJECTS')
os.environ['DJANGO_SETTINGS_MODULE'] = 'SITE.settings'

from eurasia import WSGIServer
from django.core.handlers.wsgi import WSGIHandler
server = WSGIServer(wsgiHandler()) # 启动为 fcgi
server.run()
\end{verbatim}

\subsection{启动 cherrypy}

\begin{verbatim}
#!/usr/bin/python2.5
#-*- coding: utf-8 -*-
# cherrypy 服务器启动脚本

import cherrypy
from eurasia import WSGIServer

class HelloWorld:
	def index(self):
		return 'hello world!'

	index.exposed = True

app = cherrypy.tree.mount(HelloWorld())

server = WSGIServer(app) # 启动为 fcgi
server.run()
\end{verbatim}

\subsection{混用多个框架}

eurasia.cgietc 提供的 wsgi 函数可以将任意 WSGI application 转换成普通的 handler。

handler 可以用于 config 函数。

\begin{verbatim}
from eurasia.cgietc import wsgi
config(handler=wsgi(app), port=8080)
\end{verbatim}

既然 handler 本质上是一个接受一个 httpfile 参数的处理函数, 也可以用于多框架混用。

在这个例子中, 我们将混合使用 django 和 cherrypy (以及 eurasia)。

\begin{verbatim}
#!/usr/bin/python2.5
#-*- coding: utf-8 -*-
from eurasia.cgietc import wsgi
from eurasia import config, mainloop

# django handler

import sys, os
sys.path.insert(0, '/PATH/TO/DJANGO/PROJECTS')
os.environ['DJANGO_SETTINGS_MODULE'] = 'SITE.settings'
from django.core.handlers.wsgi import WSGIHandler

django_app = WSGIHandler()
django_handler = wsgi(django_app)

# cherrypy handler

import cherrypy
class HelloWorld:
	def index(self):
		return 'hello cherrypy!'

	index.exposed = True

cherrypy_app = cherrypy.tree.mount(HelloWorld())
cherrypy_handler = wsgi(cherrypy_app)

# eurasia + django + cherrypy

def handler(httpfile):
	script_name = httpfile.script_name
	if   script_name[:8 ] == '/django/':
		# 去掉 uri 前面 '/django' 部分, 用新地址作为根路径调用 django
		httpfile.script_name = script_name[7:]
		return django_handler(httpfile)

	elif script_name[:10] == '/cherrypy/':
		# 去掉 uri 前面 '/cherrypy' 部分, 用新地址作为根路径调用 cherrypy
		httpfile.script_name = script_name[9:]
		return cherrypy_handler(httpfile)

	else:
		httpfile.write('I AM EURASIA!')
		return httpfile.close()

config(handler=handler, port=8080)
mainloop()
\end{verbatim}

\note{修改 httpfile 对象的 request_uri、script_name、query_string
这三个属性是联动的, 他们也会同时影响到 httpfile.environ 中的
REQUEST_URI、SCRIPT_NAME、QUERY_STRING 这三个环境变量 (但 httpfile.uri 不受影响)。
httpfile.environ 即是 WSGI 环境变量。}

\note{更深层次的混用, 因为 httpfile 对象被设置在 WSGI 环境中 (envrion['eurasia.httpfile']),
你可以在任何支持 WSGI 的框架中取出 httpfile 对象。
有了 httpfile 就有了 Form、SimpleUpload、Browser ... eurasia 的大部分功能,
这意味着你可以在那些框架中直接获得 eurasia 的功能。
至于如何从 WSGI 环境中得到 eurasia.httpfile, 请参考各框架的具体实现。}

\section{TCP 服务器设计}

除了 http 服务器之外, eurasia 还可以用于设计 tcp 服务器。

下面是使用 eurasia 编写的一个简单的 echo 服务器。

\begin{verbatim}
#!/usr/bin/python2.5
#-*- coding: utf-8 -*-
from eurasia import config, mainloop

# echo 服务器
def handler(tcpfile):
	while True:
		data = tcpfile.readline(1024) # 从 socket 读取
		if data == 'quit':
			tcpfile.close()
			break
		else:
			tcpfile.write(data)   # 输出到 socket

config(tcphandler=handler, port=8080) # 这里使用 tcphandler
mainloop()
\end{verbatim}

可以看到, 基本上使用 eurasia 编写 tcp 服务器与编写 http 服务器区别不大。

这里 tcp 服务器的 tcpfile 并不对 socket 中的内容作多余的处理,
你可以得到完整的 socket 输入, 并原样输出到 socket 客户端。

另一个区别在于 tcp 在 config 函数中是使用 tcphandler 指定的而不是原来的 handler。

在 eurasia 中编写 tcp 服务器甚至要比编写一个 http 服务器更简单,
eurasia 的 tcp 接口很少, 只涉及到 tcpfile 的输入输出。下面是在 tcp 服务器中可用的 tcpfile 属性。

\begin{tableii}{c|l}{}{ 属性 }{ 信息 }
\lineii{ tcpfile.pid            }{ 当前 socket 连接的 fileno }
\lineii{ tcpfile.read(size)     }{ 从 socket 中读取指定大小的内容, 默认读取全部内容 }
\lineii{ tcpfile.readline(size) }{ 行读取 }
\lineii{ tcpfile.write(data)    }{ 写 socket }
\lineii{ tcpfile.close()        }{ 关闭 socket 连接 }
\end{tableii}

这样, 基于 eurasia 的 tcp 接口, 我们现在可以方便地设计出 ftp、smtp 等基于 tcp 协议的服务器。

\section{使用标准模板}

eurasia.template 模块是 Mako 模板的一个简化版, 带有大部分 Mako 模板的功能, 和相同的标签语法。
同时 eurasia 标准模板也是编译型的模板。

template 包涵以下标签语法。

\begin{itemize}

\item \strong{表达式替换}

\begin{verbatim}
this is x: ${x}
\end{verbatim}

进一步, 取值表达式可以嵌入 python 代码, 并替换为 python 表达式的运算结果。

\begin{verbatim}
${int(a) + int(b)}
\end{verbatim}

\item \strong{控制结构}

我们可以在模板中使用条件表达式及叠代循环表达式。

这里是条件表达式。

\begin{verbatim}
%if x==1:
	x is ${x}
%elif x==2:
	x is ${x}
%else:
	x is ${x}
%endif
\end{verbatim}

这里是循环。

\begin{verbatim}
%for a in ['one', 'two', 'three', 'four', 'five']:
	%if a[0] == 't':
		its two or three
	%elif a[0] == 'f':
		four/five
	%else:
		one
	%endif
%endfor
\end{verbatim}

\item \strong{python 代码}

可以在模板中运行 python 代码。

\begin{verbatim}
<% # 这里是 Python 代码
a = 1
b = 2
%>

测试一下 a + b : ${a + b}
\end{verbatim}

\note{第一行 python 代码的缩进会被忽略, 
所以请避免使用 def、class、if、for 等依赖缩进的 python 代码。
函数中的 python 代码块在当前函数中可见, 函数外的代码块在当前模板中全局可见。}

\item \strong{函数标签 <\%def name="..." \%>}

函数是模板中最基本的调用单位, 类似于 Python 中的函数。

\begin{verbatim}
<%def name="myfunc(x)">
    this is myfunc, x is ${x}
</%def>

调用: ${myfunc(7)}
\end{verbatim}

\item \strong{<\%call expr="..."\%> 标签}

call 标签用于调用 <\%def\%> 标签,可传递额外的内嵌内容。稍后介绍。

\end{itemize}

\subsection{使用模板}

为了演示 eurasia 标准模板的用法, 这里将建立一个模板的应用范例。

在这个例子里, 我们首先将编写一个字符串模板, 带有 test1 和 test2 两个函数。
然后使用 template 模块中的 Template 函数将字符串模板编译成可执行的 python 模块,
并对模板中的 test1、test2 函数进行调用。

\begin{verbatim}
#!/usr/bin/python2.5
#-*- coding: utf-8 -*-
from eurasia.template import Template

s = '''\                  # 字符串模板
<%def name="test1(a)">
	test1, a is ${a}
</%def>

<%def name="test2(b)">
	test2, b is ${b}
</%def>'''

tmpl = Template(s)        # 将字符串模板编译成可执行的 Python 模块

print tmpl.test1('hello') # 调用模板中的 "test1" 函数
print tmpl.test2('world') # 调用模板中的 "test2" 函数
\end{verbatim}

结果是

\begin{verbatim}
test1, a is hello
test2, b is world
\end{verbatim}

<\%def> 标签定义了模板的调用接口, 在这个例子中, 我们定义了 test1 和 test2 两个调用接口。

\note{我们已经确认 CPython 中存在的一个 bug 将导致 Template(s).test1('hello') 这种写法报错。
因此, 在 CPython 中需要分开来写成 tmpl = Template(s); tmpl.test1('hello')。
也就是上面这个例子中的写法。}

\subsection{一个更为复杂的例子}

本例中, 将用到大量模板常用语法。包括有条件判断, 循环等。

\begin{verbatim}
#!/usr/bin/python2.5
#-*- coding: utf-8 -*-
from eurasia.template import Template

s = '''\
<%def name="main(lst)">
--------------------------------------------
<% # python 代码块示范
class Foo:
	def test(self):
		return 'from foo.test()'

foo = Foo()

# 在 python 代码块中可以使用 write 函数进行内容输出
write('\\n' + foo.test())

%>

${Foo().test()}

--------------------------------------------
# 循环判断示范
%for i in lst:
	%if i == 1:
		${1+2}
	%elif i == 2:
		${'hello' + 'world'}
	%else:
		${i}
	%endif
%endfor

--------------------------------------------
# 嵌入式函数示范, 只在 main 函数中可见
<%def name="bar()">
	this is bar
</%def>
${bar()}

--------------------------------------------
</%def>'''

tmpl = Template(s)
print tmpl.main([1, 2, 3, 4, 5])
\end{verbatim}

\subsection{使用 <\%call> 标签定义宏}

<\%call> 标签提供了一种较为高级的功能, 也就是模板宏。

\begin{verbatim}
<%def name="macros(a, b)">
${caller.slot1(a)}
${caller.slot2(b)}
</%def>

----------------------------

<%def name="main()">
<%call expr="macros(1, 2)">
  <%def name="slot1(a)">
    slot1, ${a}
  </%def>

  <%def name="slot2(b)">
    slot1, ${b}
  </%def>
</%call>
</%def>
\end{verbatim}

这里首先使用 <\%def> 定义了一个名为 macros 的模板宏。

然后在 main 函数中使用 <\%call> 调用 macros, 其中定义了 slot1 和 slot2,
在 macros 中可以通过 caller 取出。

该模板的入口函数是 main()。调用结果是

\begin{verbatim}
slot1, 1
slot2, 2
\end{verbatim}

\note{本例中在同一个模板中同时包涵了模板宏和宏调用, 通常情况下, 我们会将模板宏保存入外部文件。}

\note{call 只能在函数中 (<\%def>) 调用。如果把 call 放置在模板顶层, 模板在编译时会忽略 call 调用。}

\subsection{缓存编译结果}

eurasia 标准模板是一种编译型模板, 字符串模板经过 Template 函数可以编译成可执行的 python 模块。

我们也可以仅仅把字符串模板转换成 python 源代码, 在文件中保存起来,
这样可以在下次使用时直接 import 进来, 省去了编译过程, 尽管这花不了多少时间。

这里我们将用到 template 的 compile 工具。

\begin{verbatim}
#!/usr/bin/python2.5
#-*- coding: utf-8 -*-
from eurasia.template import compile

s = '''\
<%def name="main()">
	hello world!
</%def>

# 得到 python 源码
code = compile(s)

# 保存为 python 模块文件
fd = open('cache.py', w)
fd.write(code)
fd.close()

# 以模块方式导入模板
import cache
print cache.main()
\end{verbatim}

\subsection{<\%namespace> 标签}

Euraisa 标准模板中没有 Mako 中 <\%namespace> 标签的等价物, 这个功能相当于 Python 中的 import。
<\%namespace> 需要一系列关于导入路径的配置, 这比较复杂, 因此 Eurasia 换了一种方式。

Template 函数第二个参数可以设定模板中可见的环境。你可以在这里预先导入一些模板中用得到的东西, 比如宏。

这个例子在讲解 <\%call> 时出现过, 这里把宏和调用部分拆离开来。

\begin{verbatim}
#!/usr/bin/python2.5
#-*- coding: utf-8 -*-
from eurasia.template import Template
s1 = '''\ # 定义模板宏
<%def name="macros(a, b)">
${caller.slot1(a)}
${caller.slot2(b)}
</%def>'''

s2 = '''\ # 调用模板宏
<%def name="main()">
<%call expr="macros(1, 2)">
  <%def name="slot1(a)">
    slot1, ${a}
  </%def>

  <%def name="slot2(b)">
    slot1, ${b}
  </%def>
</%call>
</%def>'''

tmpl1 = Template(s1)

# 使用 env 参数指定模板环境
tmpl2 = Template(s2, env={'macros': tmpl1.macros})

tmpl2.main()
\end{verbatim}

\section{使用对象数据库}

eurasia 带有一种极其轻便的嵌入式对象数据库 shelve2。
如果用户具有 zodb 或者 durus 等对象数据库的使用经验,
会有助于理解 shelve2 的使用方法和工作方式, 但是这不是必须的, 因为 shelve2.py 要简单得多。

\begin{verbatim}
#!/usr/bin/python2.5
#-*- coding: utf-8 -*-
from eurasia.shelve2 import open

# 打开名为 test.fs 的数据库文件
d = open('test.fs')
d['foo'] = {'a': 1, 'b': 2, 'c': 3}

#d.sync() # 手动保存
d.close() # 数据库关闭时会自动保存

# 再次打开
d2 = open('test.fs')
print d2['foo']
\end{verbatim}

\subsection{对象数据库}

在对象数据库中, python 对象可以直接保存在数据库中, 而无须像关系数据库那样经过 "对象关系映射"。

下面, 我们将定义一个 MyObject 对象并存入 shelve2 数据库。

\begin{verbatim}
#!/usr/bin/python2.5
#-*- coding: utf-8 -*-
from eurasia.shelve2 import open

class MyObject:
        def __init__(self):
                self.a = None

        def print_a(self):
                print self.a

d = open('test.fs')
d['myobj'] = MyObject()
d['myobj'].a = 'hello world!'
d.close()

# 注意, 在打开数据库时, 对象定义 class MyObject 必须存在
# 否则会得到 AttributeError 的异常
d2 = open('test.fs')
d2['myobj'].print_a()
\end{verbatim}

\subsection{使用 Persistent 创建持久对象结点}

\begin{verbatim}
#!/usr/bin/python2.5
#-*- coding: utf-8 -*-
from shelve2 import open
from shelve2 import Persistent

d = open('test.fs')
obj = Persistent() # 创建持久对象结点,
obj.a = 'hello'
obj.b = 'world'
d['obj'] = obj
d.close()

d2 = open('test.fs')
obj = d2['obj']
print '%s %s!' %(obj.a, obj.b)
\end{verbatim}

在对象数据库中, Persistent 不同于一般的 Object。
他们的区别在于在 Persistent 对象和普通 Object 同时作为父对象的属性时,
当父对象被展开、取出对象数据库时, 作为父对象属性的常规 Object 也会被自动展开读取到内存中;
而 Persistent 对象在父对象展开时仅以指针形式存在, 不会被展开到内存中。
所以 Persistent 对象更节省内存并节约 IO, 这使得 Persistent 对象非常适合作为数据节点。

\subsection{定制 Persistent 对象}

自定义 Persistent 对象需从 Persistent 类继承。

\begin{verbatim}
#!/usr/bin/python2.5
#-*- coding: utf-8 -*-
from shelve2 import open
from shelve2 import Persistent

class Foo(Persistent):
	def __init__(self, a, b):
		self.a = a
		self.b = b

	def print_ab(self):
		print '%s %s!' %(self.a, self.b)

d = open('test.fs')
d['foo'] = Foo('hello', 'world')
d.close()

d2 = open('test.fs')
d2['foo'].print_ab()
\end{verbatim}

\subsection{使用 BTree 创建大容量数据结点}

BTree 对象可以用以保存数以千万计的对象, 同时被保存的这些对象是按照 Key 排序的, 你可以以任何方式定位、遍历这些内容。

\begin{verbatim}
#!/usr/bin/python2.5
#-*- coding: utf-8 -*-
from shelve2 import lazy as open
from shelve2 import BTree

d = open('test.fs')
tr = BTree()
for i in xrange(1000):
	tr[i] = `i`

d['tr'] = tr
d.close()

d2 = open('test.fs')
print d2['tr']
\end{verbatim}

BTree 除了支持所有类似于 dict 的方法外, 还支持各种遍历方法:

\begin{tableii}{c|l}{}{ 说明 }{ 属性 }
\lineii{ 正向遍历                 }{ BTree.iter() }
\lineii{ 反向遍历                 }{ BTree.reversed() }
\lineii{ 正向遍历返回键、值 }{ BTree.iteritems() }
\lineii{ 反向遍历返回键、值 }{ BTree.items_backward() }
\lineii{ 正向遍历返回键、值 }{ BTree.items_from(key, closed=True) }
\lineii{ 反向遍历返回键、值 }{ BTree.items_backward_from(key, closed=True) }
\lineii{ 遍历区域返回键、值 }{ BTree.items_range(start, end, closed_start=True, closed_end=False) }
\end{tableii}

BTree 对象可以以和 Persistent 对象类似的方法派生。

shelve2 对象数据库的根对象就是一个 BTree 对象。 

\subsection{Persistent 和 BTree}

在 zope/plone 中, 不仅文件是 Persistent 对象, 连文件夹也是 Persistent 对象,
此时一个文件就是 Persistent 对象的一个属性。
只有在文件夹过大的时候我们才会使用从 BTree 派生出来的 PloneLargeFolder。
所以从本质上说, Persistent 和 BTree 都是容器对象,
他们的区别在于 Persistent 对象的内容在数据库里是连续存储,
而在内存中则变成 Hash Table, 在展开 Persistent 对象时其所有的属性也会随之一次性全部展开;
而 BTree 对象无论在数据库中还是内存中都是以平衡树的方式存储的,
读取 BTree 对象时 BTree 对象中的内容不会一次完全展开, 而是按需要分批展开。

所以在需要频繁读取容器内容时、并且容器中放置的内容并不多的情况下 Persistent 会快于 BTree。
而当保存的内容很多时 BTree 要更快。但是我们往往很难知道某个容器将来到底会扩充到何种规模,
所以可以在任何时候都使用 BTree 作为容器, 在性能上这其实并无大碍。

\section{配置文件支持}

eurasia.pyetc 模块为配置文件的读取提供了支持。eurasia 认为使用 python 语法来编写配置文件是个好主意。

pyetc 中的 load(filename, env={}) 函数能读取指定的 python 源码文件 (文件后缀并不需要一定是 ".py"), 并设定配置文件中的可见环境 env。load 函数能将指定源码文件转换成 python 模块并返回。

我们首先编写一个名为 \file{httpd.conf} 的配置文件。

\begin{verbatim}
Server(controller='Products.default.controller'
	port=8080)

user = 'nobody'
\end{verbatim}

下面是对于这个配置文件的解析

\begin{verbatim}
from eurasia import pyetc
config = {}
def Server(**args):
	config.update(args)

mod = pyetc.load('httpd.conf', env={Server:Server})
print 'Server:', config
print 'user', mod.user
\end{verbatim}

\note{pyetc 模块可以被用于导入执行任意指定路径和文件后缀的 python 模块}


\chapter{API 文档}

\section{eurasia}

\begin{itemize}

\item \strong{config(**args)}

config() 参数可以接受如下参数

\begin{tableii}{c|l}{}{ 参数 }{ 说明 }
\lineii{ handler     }{ 指定一个服务器 handler, 如果指定地址, 则为  http 服务器, 否则即为 fcgi 服务器 }
\lineii{ controller  }{ 同 handler }
\lineii{ httphandler }{ 指定一个 http 服务器的 handler, 如果没有指定地址, 默认使用 0.0.0.0:8080 }
\lineii{ fcgihandler }{ 指定一个 fcgi 服务器的 handler, 忽略地址设定 }
\lineii{ tcphandler  }{ 指定一个 tcp 服务器的 handler, 需要设定地址 }
\lineii{ port }{ 指定服务器 port, 比如 8080 }
\lineii{ bind }{ 指定多个服务器地址, 比如 '192.168.0.1:8080, 192.168.0.2:80' }
\end{tableii}

\note{只能指定一个 handler, port 和 bind 不能同时指定}

\item \strong{mainloop(cpus=False)}

启动由 config() 设置的所有服务器, 应用全局 verbose、uid、procname 设置。

如果 cpus 参数默认 False, 使用单进程。如果 cpus 为 True 则启动本机 cpu (及 cpu 核心) 数量的进程。
你也可以直接把 cpus 设定为某一整数, 手工指定进程数量。

\note{在多进程模式下, 全局变量不是进程间共享的, 你需要使用诸如 dbus 等进程间通信手段。}

\note{fcgi 服务器会忽略 cpus 设定, 总是使用单进程。
事实上, 诸如 apache 这类 http 服务器会启动多个 fcgi 服务器进程, 亦即 fcgi 服务器总是多进程的。}

\item \strong{WSGIServer(application, port=None, bind=None, bindAddress=None)}

返回一个 WSGIServer 对象。

你可以通过 port、bind、bindAddress 之一指定端口、多个地址、单个地址。
port、bind 的设置与 config() 函数一样, bindAddress 参数是为了与 flup.WSGIServer 相兼容。

你可以通过 WSGIServer 对象的 run(cpus=False) 方法或者 server_forever(cpus=False) 方法来启动服务器。

\item \strong{verbose}

如果设置了 eurasia.verbose, 并且取值为 False,
那么通过 eurasia.mainloop() 启动服务器, 会禁止一切调试输出。

\note{该项设置对 eurasia.web.mainloop()、eurasia.fcgi.mainloop() 无效}

\item \strong{uid}

如果设置了 eurasia.uid, 那么通过 eurasia.mainloop() 启动服务器,
会使用指定用户。你可以直接指定用户名。

\note{使 uid 生效需要 root 权限}

\note{该项设置对 eurasia.web.mainloop()、eurasia.fcgi.mainloop() 无效}

\item \strong{user}

与 eurasia.uid 相同, 但是你不能同时设置 eurasia.uid 和 eurasia.user。

\item \strong{procname}

如果设置了 eurasia.procname, 那么通过  eurasia.mainloop() 启动服务器,
会使用指定的进程名。

\note{该项设置对 eurasia.web.mainloop()、eurasia.fcgi.mainloop() 无效}

\item \strong{libc}

如果设置了 eurasia.procname, 那么通过  eurasia.mainloop() 启动服务器, 会使用指定的进程名。
这需要用到 libc, 大部分情况下 eurasia 的 libc 默认值 "/lib/libc.so.6" 都可以工作,
但是在少数情况诸如嵌入式系统上, 你需要设置 libc 为正确的位置, 比如 "/lib/libuClibc-0.9.27.so"。

\note{如果没有设置 eurasia.procname, 该项设置会被忽略}

\note{该项设置对 eurasia.web.mainloop()、eurasia.fcgi.mainloop() 无效}

\end{itemize}

\section{eurasia.socket2}

\begin{itemize}

\item \strong{config(**args)}

config() 参数可以接受如下参数

\begin{tableii}{c|l}{}{ 参数 }{ 说明 }
\lineii{ handler     }{ 指定一个 tcp 服务器 handler }
\lineii{ controller  }{ 同 handler }
\lineii{ tcphandler  }{ 同 handler }
\lineii{ port }{ 指定服务器 port, 比如 8080 }
\lineii{ bind }{ 指定多个服务器地址, 比如 '192.168.0.1:8080, 192.168.0.2:80' }
\end{tableii}

\note{只能指定一个 handler, port 和 bind 不能同时指定}

\item \strong{mainloop(cpus=False)}

启动由 socket2.config() 设置的所有服务器。

如果 cpus 参数默认 False, 使用单进程。如果 cpus 为 True 则启动本机 cpu (及 cpu 核心) 数量的进程。
你也可以直接把 cpus 设定为某一整数, 手工指定进程数量。

\note{在多进程模式下, 全局变量不是进程间共享的, 你需要使用进程间通信。}

\item \strong{SocketFile 对象}

SocketFile 对象是由 TcpServer 生成并传递给 tcphandler 的。

SocketFile 对象的属性如下。

\begin{tableii}{c|l}{}{ 属性/方法 }{ 说明 }
\lineii{ socket   }{ 原始 socket }
\lineii{ pid      }{ socket 的 fileno }
\lineii{ address  }{ socket 地址, 通常用于获取客户端地址, 取值为 (ip, port) }
\lineii{ fileno() }{ 取得 socket 的 fileno }
\lineii{ read(size=-1)     }{ 从  socket 读取指定大小的数据, 默认读取所有 }
\lineii{ readline(size=-1) }{ 行读取 }
\lineii{ write(data) }{ 向 socket 写入数据 }
\lineii{ close() }{ 关闭 socket }
\end{tableii}

\note{在 read()/readline()/write()/close() 等 IO 操作中如果客户端已经断开连接则会抛出 IOError}

\item \strong{TcpHandler(handler)}

声明 handler 为 tcphandler。

\begin{verbatim}
@TcpHandler
def handler(sockfile):
	...
\end{verbatim}

\item \strong{TcpServerSocket(address)}

返回一个绑定到  address 的 socket。

\item \strong{TcpServer(sock, handler)}

创建一个绑定到 sock 使用 handler 的 tcp 服务器, 可以通过 mainloop() 启动该服务器。

下面是一个完整的例子。

\begin{verbatim}
from eurasia.socket2 import TcpServer, TcpServerSocket, TcpHandler
def handler(sockfile):
	sockfile.write('hello world!')
	sockfile.close()

# config(tcphandler=handler, port=8080)
TcpServer(TcpServerSocket('0.0.0.0', 8080), TcpHandler(handler))

mainloop()
\end{verbatim}

\end{itemize}

\section{eurasia.web}

\begin{itemize}

\item \strong{config(**args)}

config() 参数可以接受如下参数

\begin{tableii}{c|l}{}{ 参数 }{ 说明 }
\lineii{ handler     }{ 指定一个 http 服务器 handler }
\lineii{ controller  }{ 同 handler }
\lineii{ httphandler }{ 同 handler }
\lineii{ tcphandler  }{ 指定一个 tcp 服务器 handler }
\lineii{ port }{ 指定服务器 port, 比如 8080 }
\lineii{ bind }{ 指定多个服务器地址, 比如 '192.168.0.1:8080, 192.168.0.2:80' }
\end{tableii}

\note{只能指定一个 handler, port 和 bind 不能同时指定}

\item \strong{mainloop(cpus=False)}

启动由 web.config() 设置的所有服务器。

如果 cpus 参数默认 False, 使用单进程。如果 cpus 为 True 则启动本机 cpu (及 cpu 核心) 数量的进程。
你也可以直接把 cpus 设定为某一整数, 手工指定进程数量。

\note{在多进程模式下, 全局变量不是进程间共享的, 你需要使用进程间通信。}

\item \strong{HttpFile 对象}

HttpFile 对象是由 TcpServer 生成并传递给 httphandler 的。

HttpFile 对象的属性如下。

\begin{tableii}{c|l}{}{ 属性/方法 }{ 说明 }
\lineii{ sockfile   }{ 原始 socket 的 SocketFile 封装, 由服务器传递给 httphandler }
\lineii{ pid        }{ sockfile 的 fileno }
\lineii{ address    }{ socket 地址, 通常用于获取客户端地址, 取值为 (ip, port) }
\lineii{ uid        }{ 读取和指定用户的持久 id }
\lineii{ environ    }{ 一个 dict, 存放 cgi 环境变量, 也是 WSGI 环境变量 }
\lineii{ request_uri  }{ 读取和设置 REQUEST_URI 环境变量 (存放于 httpfile.environ), 会影响到 script_name、query_string }
\lineii{ script_name  }{ 读取和设置 SCRIPT_NAME 环境变量, 会影响到 request_uri }
\lineii{ query_string }{ 读取和设置 QUERY_STRING 环境变量, 会影响到 request_uri }
\lineii{ __getitem__(key) }{ 读取请求的头部 }
\lineii{ __setitem__(key) }{ 设置响应的头部 }
\lineii{ fileno()   }{ 取得 socket 的 fileno }
\lineii{ read(size=-1)     }{ 从  http 请求读取指定大小的数据, 默认读取所有 }
\lineii{ readline(size=-1) }{ 行读取 }
\lineii{ write(data)       }{ 向 http 客户端写入数据 }
\lineii{ close() }{ 结束本次请求 }
\lineii{ begin() }{ 发送 http 头, 并进入长连接流模式 }
\lineii{ wbegin(data) }{ 发送 http 头和 data, 并进入流模式, 相当于 begin() + write(data) }
\end{tableii}

\note{在 read()/readline()/write()/close() 等 IO 操作中如果客户端已经断开连接则会抛出 IOError}

\item \strong{HttpHandler(handler)}

声明 handler 为 httphandler。

\begin{verbatim}
@HttpHandler
def handler(httpfile):
	...
\end{verbatim}

\item \strong{TcpServerSocket(address)}

返回一个绑定到  address 的 socket。

\item \strong{TcpServer(sock, handler)}

创建一个绑定到 sock 使用 handler 的 tcp 服务器, 可以通过 mainloop() 启动该服务器。

下面是一个完整的例子。

\begin{verbatim}
from eurasia.web import TcpServer, TcpServerSocket, HttpHandler
def handler(httpfile):
	httpfile.write('hello world!')
	httpfile.close()

# config(httphandler=handler, port=8080)
TcpServer(TcpServerSocket('0.0.0.0', 8080), HttpHandler(handler))

mainloop()
\end{verbatim}

\item \strong{Form(httpfile, max_size=1048576)}

表单处理, 见 eurasia.cgietc.SimpleUpload

\item \strong{SimpleUpload(httpfile)}

上传文件处理, 见 eurasia.cgietc.SimpleUpload

\item \strong{Browser(httpfile)}

远程 js 调用, 见 eurasia.cgietc.Browser

\item \strong{json(object)}

将 python 基础对象转换成 json 报文。仅支持 dict、list、int、long、float、basestring。

\note{非 ascii 字符必须是 unicode 而非 str}

\item \strong{wsgi(application)}

将 WSGI application 转换成 handler。

\end{itemize}

\section{eurasia.fcgi}

eurasia.web 的 fastcgi 克隆

\begin{itemize}

\item \strong{config(**args)}

config() 参数可以接受如下参数

\begin{tableii}{c|l}{}{ 参数 }{ 说明 }
\lineii{ handler     }{ 指定一个 fcgi 服务器 handler }
\lineii{ controller  }{ 同 handler }
\lineii{ fcgihandler }{ 同 handler }
\lineii{ port }{ 指定服务器 port, 比如 8080 }
\lineii{ bind }{ 指定多个服务器地址, 比如 '192.168.0.1:8080, 192.168.0.2:80' }
\end{tableii}

\note{只能指定一个 handler, port 和 bind 不能同时指定}

\item \strong{mainloop(cpus=False)}

启动由 fcgi.config() 设置的所有服务器。

如果 cpus 参数默认 False, 使用单进程。如果 cpus 为 True 则启动本机 cpu (及 cpu 核心) 数量的进程。
你也可以直接把 cpus 设定为某一整数, 手工指定进程数量。

\note{在多进程模式下, 全局变量不是进程间共享的, 你需要使用进程间通信。}

\item \strong{FcgiFile 对象}

FcgiFile 对象是由 TcpServer 生成并传递给 fcgihandler 的。

FcgiFile 的属性与 HttpFile 相同。

\note{FcgiFile 可以通过调用 begin() 进入长连接流模式, 但是会阻塞掉该 fcgi 进程, 因此不推荐使用}

\item \strong{FcgiHandler(handler)}

声明 handler 为 fcgihandler。

\begin{verbatim}
@FcgiHandler
def handler(httpfile):
	...
\end{verbatim}

\item \strong{FcgiServerSocket()}

返回一个 fcgi 服务器 socket。

\item \strong{TcpServer(sock, handler)}

创建一个绑定到 sock 使用 handler 的 tcp 服务器, 可以通过 mainloop() 启动该服务器。

下面是一个完整的例子。

\begin{verbatim}
from eurasia.fcgi import TcpServer, FcgiServerSocket, FcgiHandler
def handler(httpfile):
	httpfile.write('hello world!')
	httpfile.close()

# config(fcgihandler=handler)
TcpServer(FcgiServerSocket(), FcgiHandler(handler))

mainloop()
\end{verbatim}

\item \strong{Form(httpfile, max_size=1048576)}

表单处理, 见 eurasia.cgietc.SimpleUpload

\item \strong{SimpleUpload(httpfile)}

上传文件处理, 见 eurasia.cgietc.SimpleUpload

\item \strong{Browser(httpfile)}

远程 js 调用, 见 eurasia.cgietc.Browser

\item \strong{json(object)}

将 python 基础对象转换成 json 报文。仅支持 dict、list、int、long、float、basestring。

\note{非 ascii 字符必须是 unicode 而非 str}

\item \strong{wsgi(application)}

将 WSGI application 转换成 handler。

\end{itemize}

\section{eurasia.cgietc}

\begin{itemize}

\item \strong{Form(httpfile, max_size=1048576)}

Form 函数将用户提交的表单解析成一个 dict 对象。

你可以通过设定 max_size 来限制用户 POST 数据的大小, 如果用户提交的数据超过限制, Form 函数会抛出 IOError。

\note{如果用户提交了多个重名的表单变量, Form 会使用 list 来保存。比如 "?a=1\&a=2\&a=3\&b=4", form['a'] 的取值是 ['1', '2', '3'], 而 form['b'] 的取值是 '4'。}

\item \strong{SimpleUpload(httpfile)}

SimpleUpload 会返回一个上传文件的文件描述符 (fd), 我们可以从 fd.filename 和 fd.size 中取出文件名和文件大小。
其余的表单变量可以通过 fd[key] 取出。

\begin{tableii}{c|l}{}{ 属性 }{ 信息 }
\lineii{ SimpleUpload.size           }{ 上传文件的大小 }
\lineii{ SimpleUpload.filename       }{ 上传文件的文件名 }
\lineii{ SimpleUpload[key]           }{ 除了上传文件之外, 其余表单变量 }
\lineii{ SimpleUpload.read(size)     }{ 从上传文件中读取指定大小的内容, 默认读取全部内容 }
\lineii{ SimpleUpload.readline(size) }{ 行读取 }
\end{tableii}

\note{SimpleUpload 要求表单变量名 (这里是 'a') 必须小于文件控件的名称 (这里是 'z-file')。
因为对 SimpleUpload 组件来说, 文件组件的名字并不重要, 所以请尽量起成 "zzzzzzzz" 之类的名字,
以保证他的值是最大的。另外, SimpleUpload 组件只支持上传一个文件。
当需要提交多个文件时, 你可以使用多个 SimpleUpload 组件。}

\note{SimpleUpload 并不会真的把用户上传的文件建立在磁盘上, 也不会把用户的上传读取到内存中,
而是在调用 read() 时才会从 socket 中去获取。因此, 在没有读取完整个上传文件之前,
上传用户的浏览器将一直保持在 "正在上传" 的状态下。}

\note{http 协议并不是每次都确切地知道用户上传文件的大小,
SimpleUpload 能够猜测出用户上传内容的大小, 对于标准的浏览器和客户端这通常是准确的。
在极少数情况下 SimpleUpload.size 和 SimpleUpload.read 取出的内容大小并不匹配,
这时应以 SimpleUpload.read 取出的实际内容大小为准, 一般来说, 你可以认为这是一种恶意行为。}

\note{在向用户发送任何内容之前, 请先确保已经完全读取了上传文件, 否则会发生读写错误 (IOError)。}

\note{在读取文件时如果用户已经断开连接, read 会抛出 IOError。}

\item \strong{Browser(httpfile, domain=None)}

取得浏览器对象, 可以直接调用远程浏览器上的 javascript 函数。

我们可以在 Brwoser 对象创建时通过 domain 指定 js 域 (document.domain),
方便 javascript 的跨子域调用。

\note{如果不指定 domain, Browser 对象会指定 domain 为顶级域名,
如果当前域为 "www.example.com" 那么默认即为 "example.com"}

\item \strong{json(object)}

将 python 基础对象转换成 json 报文。仅支持 dict、list、int、long、float、basestring。

\note{非 ascii 字符必须是 unicode 而非 str}

\item \strong{wsgi(application)}

将 WSGI application 转换成 handler。

\end{itemize}

\section{eurasia.shelve2}

\begin{itemize}

\item \strong{open(filename, pool=None)}

返回数据库连接对象。如果指定了 pool 则使用指定的线程池 (Pool)。

数据库连接对象具有 BTree 所有属性和方法, 用于操作根对象。

\begin{tableii}{c|l}{}{ 属性 }{ 信息 }
\lineii{ sync() }{ 保存所有的修改 }
\lineii{ close() }{ 保存所有修改并关闭连接 }
\end{tableii}

\item \strong{Pool(n=16)}

返回线程池对象。使用指定的线程数量 (n)。

\item \strong{Persistent 类}

Persistent 定义了对象数据库的持久对象。
可以增加、删除、修改 Persitent 对象的属性, 在调用数据库的保存接口时这些修改会自动提交。

\item \strong{BTree 类}

BTree 定义了对象数据库的持久 btree (类似于 python 字典) 对象。

BTree 支持的方法有

\begin{tableii}{c|l}{}{ 属性 }{ 说明 }
\lineii{ __getitem__(key) }{ 取值  }
\lineii{ __setitem__(key, value) }{ 设置 key 的值  }
\lineii{ get(key, default=None) }{ 取值, 如果  key 不存在返回  default }
\lineii{ update(dict) }{ 批量赋值 }
\lineii{ keys() }{ 返回所有键 }
\lineii{ values() }{ 返回所有值 }
\lineii{ items() }{ 以  [(key, value), ...] 形式返回所有键值对 }
\lineii{ iter() }{ 正向遍历  }
\lineii{ reversed() }{ 反向遍历 }
\lineii{ iteritems() }{ 正向遍历返回键、值 }
\lineii{ items_backward() }{ 反向遍历返回键、值 }
\lineii{ items_from(key, closed=True) }{ 正向遍历返回键、值 }
\lineii{ items_backward_from(key, closed=True) }{ 反向遍历返回键、值 }
\lineii{ items_range(start, end, closed_start=True, closed_end=False) }{ 遍历区域返回键、值 }
\end{tableii}

\end{itemize}

\section{eurasia.template}

\begin{itemize}

\item \strong{Template(text, env={})}

解析字符串 text, 返回一个模板对象。可以通过 env 设定模板中可见的环境变量。

\item \strong{compile(text)}

将字符串 text 编译为 python 源码。

\end{itemize}

\section{eurasia.pyetc}

\begin{itemize}

\item \strong{load(fullpath, env={}, module=Module)}

导入指定路径 fullpath 的 python 源码, 返回为 python 模块。

通过设定  env 参数可以指定在导入 python 文件时可见的环境变量。

通过 module 参数可以指定返回的模块类型。

\item \strong{reload(module)}

重新加载由 pyetc.load() 导入的模块。

\item \strong{unload(module)}

卸载由 pyetc.load() 导入的模块。

\end{itemize}

\section{eurasia.utility}

\begin{itemize}

\item \strong{cpu_count()}

得到 cpu 或者 cpu 核心数。

\item \strong{setuid(user)}

修改当前程序的身份。

\item \strong{setprocname(procname, libc=None)}

修改当前进程的进程名。如果系统 libc 的位置不在 '/lib/libc.so.6' 你需要手工指定。

\note{不支持 windows}

\item \strong{dummy()}

将 stdin、stdout 以及 stderr 重定向到 /dev/null, 禁止标准输入输出。通常用于后台服务器进程。

\item \strong{daemonize(program, *args)}

在后台执行 program, 使用指定的命令行参数 args。

调用 daemonize 以后, 本进程将结束。

\end{itemize}

\section{eurasia.daemon}

\note{eurasia.daemon 模块仅限于 unix/linux 系统。}

\note{eurasia.daemon 模块依赖于 python2.5。}

\begin{itemize}

\item \strong{Daemon 类}

Daemon 接受如下初始化参数

\begin{tableii}{c|l}{}{ 参数 }{ 说明 }
\lineii{ program }{ 在后台运行的程序, 这个参数是必须的 }
\lineii{ address }{ pid 文件名 (也可以指定一个 socket 地址) }
\lineii{ verbose }{ 如果设置为 False 则禁止调试 }
\lineii{ stdout  }{ 在禁止调试的情况下, 指定 stdout 重定向某个文件句柄 }
\lineii{ stderr  }{ 在禁止调试的情况下, 指定 stderr 重定向某个文件句柄 }
\lineii{ directory }{ 指定进程工作目录 }
\lineii{ umask }{ 设置 umask }
\end{tableii}

Daemon 提供的方法

\begin{tableii}{c|l}{}{ 属性/方法 }{ 说明 }
\lineii{ __call__(*args) }{ 在后台启动 program, 使用命令行参数 args }
\lineii{ status() }{ 返回指定进程的运行状态, 'stopped' 或者 'running' }
\lineii{ stop() }{ 结束后台进程 program, 如果失败会抛出 daemon.error }
\lineii{register_function(func, name)}{ 因为 Daemon 是一个 xmlrpc 服务器, 你可以通过这个函数向 xmlrpc 服务器注册更多的方法 }
\end{tableii}

调用 Daemon 对象的  __call__ 方法以后, 将在指定的  address 建立一个 xmlrpc 服务。
xmlrpc 服务提供三个方法, 即 Daemon.status()、Daemon.stop()

\item \strong{ServerProxy 类}

通过 ServerProxy(address, **args) 可以得到指定地址 address 上的 xmlrpc 服务器。

args 参数类似 xmlrpclib.ServerProxy 的设定。

通过 ServerProxy 我们可以调用指定 address 上的 Daemon 服务器上的方法。具体请参见 xmlrpclib。

\item \strong{error 类 (异常)}

Daemon 相关的异常。

\end{itemize}

\end{document}
