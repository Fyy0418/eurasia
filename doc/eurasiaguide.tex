\documentclass{manual}

\title{Eurasia 用户指引}

\author{沈崴}

\authoraddress{
	项目: \url{http://eurasia.pyforce.com}
	邮件列表: \url{http://groups.google.com/group/eurasia-users}
}

\date{2008 年 11 月 11 日}

\release{3.0.0a5}

\makeindex

\begin{document}

\maketitle

\begin{abstract}

\noindent

Eurasia 高性能应用服务器, 同时也是一种简洁高效的开发框架, 可以用于面向高并发的长连接 HTTP 服务器, 和 TCP 服务器的设计。

\end{abstract}

\tableofcontents


\chapter{安装}

\section{系统要求}

Eurasia 基于 Python 2.5。一般来说, 支持 Python 2.5 的计算机平台都能够运行 Eurasia。

推荐配置:

\begin{itemize}

\item \strong{32 位 x86 体系结构}

目前尚没有证据证明 Eurasia 能通过 64 位或其他体系结构来获得额外的性能提升。

\item \strong{Unix 或 Linux 操作系统}

Eurasia 基于 Unix 异步 IO 设计。

\item \strong{Stackless Python 2.5.2}

Eurasia 基于 Stackless Python 轻便线程技术设计。所以 Stackless Python 是最适合的。

\end{itemize}

\subsection{兼容性话题}

\strong{Windows 操作系统}

需要注意的是, Eurasia 在 Windows 上并不能达到 Unix/Linux 上的并发性能。

\strong{Python 2.5}

普通的 Python 2.5 可以通过 greenlet 技术来实现 Eurasia 所需的 Stackless 特性。在性能上并不会有明显差别。如果你正在使用非 Stackless 的 Python 版本, Eurasia 会自动从网上下载并安装 py 包, 你需要准备好网络连接。

\strong{Python 2.6}

Python 2.6 是否能够很好地兼容 Eurasia, 未经充分测试。但无论如何 Eurasia 还是推荐使用 2.5 版本的 Python, 因为 Python 2.5 在性能和成熟度上似乎都要优于 Python 2.6。同时 Eurasia 也没有 (借助 Python 2.6 的过渡) 升级到 Python 3000 的明显愿望, 目前看来 Eurasia 更倾向于 PyPy。

\section{开始安装}

Eurasia 的官方网站是 \url{http://eurasia.pyforce.com}, 可以在这里下载到 Eurasia 的最新版本。

你可以通过解压目录中的安装脚本 \file{setup.py} 来安装 Eurasia 库:

\begin{verbatim}
    python setup.py install
\end{verbatim}

对于非 Stackless 的 Python 2.5, Eurasia 需要 py (\url{http://codespeak.net/py}) 包的支持。如果没有检查到 py 包, 安装程序会尝试通过 setuptools 自动下载并安装 py 包。

除此之外, Eurasia 还提供了一种更为便捷的绿色安装方式, 通过 \file{mkinstance.py} 脚本, 你可以创建一个直接可用的 Eurasia 站点。这时, 把 Eurasia 安装到 Python 上去并不是必须的。

\begin{verbatim}
    python mkeurinstance.py
\end{verbatim}

需要注意的是对于非 Stackless Python, py 包仍然是必须的, \file{mkinstance.py} 脚本同样会尝试下载, 并自动安装 py 包。

关于 \file{mkinstance.py} 及 Eurasia 默认的应用服务器架构, 后面会有专门章节详加叙述。


\chapter{快速开始}

我们将从最简单的 hello world 开始, 通过一些范例程序, 快速进入 Eurasia 的世界。

\section{第一个 http 服务器: hello world!}

下面我们来编写一个服务器。

\begin{verbatim}
#!/usr/bin/python2.5
from eurasia.web import config, mainloop, Response
def controller(request):
	response = Response(request)
	response.write('hello world!')
	response.close()

config(controller=controller, port=8080)
mainloop()
\end{verbatim}

保存为 \file{hello.py}, 启动服务器。

\begin{verbatim}
$/usr/bin/python hello.py
\end{verbatim}

这个程序简单地把一个处理函数 controller 绑定到本机的 8080 端口, 所有发往本机 8080 端口的 HTTP 请求都会得到一个 "hello world!" 的响应。

这样, 我们的第一个 Eurasia 程序就完成了。

\section{使用 request 和 response 对象}

在 hello world 中, 我们已经接触到了 request 和 response 对象。这里我们再进一步了解他们的使用。

\subsection{通过 request 对象访问请求信息}

controller 函数的 request 参数对用户请求进行了包装, request 对象包含以下属性。

\begin{tableii}{c|l}{}{ 属性 }{ 信息 }
\lineii{ request.path           }{ 请求地址, 比如 '/index.html' }
\lineii{ request.method         }{ 请求方法, 取值是大写的 'GET' 或大写的 'POST' }
\lineii{ request.version        }{ 请求的 HTTP 版本, 'HTTP/1.0' 或 'HTTP/1.1' }
\lineii{ request['Http-Header'] }{ HTTP 请求头部, 比如 request['User-Agent'], 注意 Key 是首字母大写的 }
\end{tableii}

你可以编写一个程序来测试这些属性。

\begin{verbatim}
#!/usr/bin/python2.5
from eurasia.web import config, mainloop, Response
from cStringIO import StringIO
def controller(request):
	out = StringIO()
	print >> out, '%s %s %s' %(request.method, request.path, request.version)
	for key, value in request.items():
		print >> out, '%s: %s' %(key, value)

	response = Response(request)
	response.write('<pre>%s</pre>' %out.getvalue())
	response.close()

config(controller=controller, port=8080)
mainloop()
\end{verbatim}

\subsection{使用 response 对象与浏览器进行通信}

response 的使用与 request 基本上是对应的。

\begin{tableii}{c|l}{}{ 属性 }{ 信息 }
\lineii{ response.status             }{ 响应状态, 常见的有 200 (正常响应, 默认), 404 (未找到) 等 }
\lineii{ response.version            }{ 响应的 HTTP 版本, 'HTTP/1.1' (默认) 或 'HTTP/1.0' }
\lineii{ response['Response-Header'] }{ HTTP 头, 比如 response['Content-Type'] = 'text/html', 注意 Key 是首字母大写的 }
\end{tableii}

你可以在创建 response 对象时直接指定 status 和 version。

\begin{verbatim}
response = Response(request, status=200, version='HTTP/1.1')
\end{verbatim}

下面是一个返回 404 Not Found 的例子, 服务器不管收到任何请求, 用户都会得到一个 404 页面。

\begin{verbatim}
#!/usr/bin/python2.5
from eurasia.web import config, mainloop, Response
page = '''\
<html>
<head>
	<title>404 Not Found</title>
</head>
<body>
	<h1>Not Found</h1>
	<p>The requested URL %s was not found on this server.</p>
	<hr>
	<address>Eurasia Demo Server</address>
</body>
</html>'''

def controller(request):
	response = Response(request, status=404)
	response['Content-Type'] = 'text/html'
	response.write(page %request.path)
	response.close()

config(controller=controller, port=8080)
mainloop()
\end{verbatim}

\subsection{一个综合应用: 文件服务器}

接下来, 是一个简单的文件服务器。接到请求后, 服务器会到系统的 \file{/www} 目录下去寻找指定的文件, 如果文件存在, 则返回文件, 如果文件不存在, 则返回一个简单的 404 未找到页面。另外, 这个简单的文件服务器并不支持列出目录, 如果发现用户请求的是一个目录, 则返回一个简单的 403 拒绝页面。

\note{需要注意的是, 这个服务器只是示范用途, 切勿实用。在实用中, Eurasia 已经为你准备了文件服务器, 请使用 \file{eurasia.cherrypy.Directory}, 这在后面会有详细介绍。}

\begin{verbatim}
#!/usr/bin/python2.5
#-*- coding: utf-8 -*-
import os.path
from eurasia.web import config, mainloop, Response

def controller(request):
	filename = '/www' + request.path
	if not os.path.exists(filename): # 文件不存在  , 返回 404
		response = Response(request, status=404)
		response.write('<h1>Not Found</h1>')
		return response.close()

	if os.path.isdir(filename):      # 请求的是目录, 返回 403
		response = Response(request, status=403)
		response.write('<h1>Forbidden</h1>')
		return response.close()

	try:
		data = open(filename).read()
	except (IOError, OSError):                  # 文件读取错误, 返回 500
		response = Response(request, status=500)
		response.write('<h1>Internal Server Error</h1>')
		return response.close()

	# 正确返回文件内容, 因为设定了 application/octet-stream 的 Content-Type,
	# 浏览器将提示用户是否下载该文件, 而不是直接显示出来
	response = Response(request)
	response['Content-Type'] = 'application/octet-stream'
	response.write(data)
	response.close()

config(controller=controller, port=8080)

mainloop()
\end{verbatim}

我们会发现, 这里仅仅用到了 hello world 这一节的知识, request 和 response 虽然简单, 但已经可以完成许多应用。

\section{表单处理与文件上传}

Eurasia 为你提供了表单处理工具, 还有处理文件上传的简单工具。

\subsection{Form}

Form 函数将用户提交的表单解析成一个 dict 对象。

\begin{verbatim}
	formdict = Form(request)
\end{verbatim}

你可以通过设定 max_size 来限制用户 POST 数据的大小, 如果用户提交的数据超过限制, Form 函数会抛出 IOError。

\begin{verbatim}
#!/usr/bin/python2.5
#-*- coding: utf-8 -*-
import os.path
from eurasia.web import config, mainloop, Form, Response

def controller(request):
	try:
		form = Form(request, max_size=1024)
	except IOError:
		request.close()

	response = Response(request)
	response['Content-Type'] = 'text/plain'
	print >> response, `form`
	response.close()

config(controller=controller, port=8080)

mainloop()
\end{verbatim}

\note{如果用户提交了多个重名的表单变量, Form 会使用 list 来保存。比如用户提交了 url?a=1\&a=2\&a=3\&b=4, 那么 form['a'] 的取值是 ['1', '2', '3'], form['b'] 的取值是 '4'。}

\subsection{文件上传}

这是一个典型的文件提交页面。

\begin{verbatim}
<html>
<form action="/simpleupload" method="post" enctype="multipart/form-data">
<input type="hidden" name="a" value="表单变量A" />
<input type="file" name="z-file" />
<input type="submit" />
</form>
</html>
\end{verbatim}

Eurasia 的 SimpleUpload 工具对文件上传提供了支持。

\begin{verbatim}
	fd = SimpleUpload(request)
\end{verbatim}

SimpleUpload 会返回一个上传文件的描述符 (fd), 我们可以从 fd.filename 和 fd.size 中取出文件名和文件大小。其余的表单变量可以通过 fd[name] 的形式取出。

\begin{tableii}{c|l}{}{ 属性 }{ 信息 }
\lineii{ SimpleUpload.size           }{ 上传文件的大小 }
\lineii{ SimpleUpload.filename       }{ 上传文件的文件名 }
\lineii{ SimpleUpload['name']        }{ 除了上传文件之外, 表单中其余的变量 }
\lineii{ SimpleUpload.read(size)     }{ 从上传文件中读取指定大小的内容, 默认读取全部内容 }
\lineii{ SimpleUpload.readline(size) }{ 行读取 }
\end{tableii}

下面是完整的处理脚本。服务器将返回上传文件的缩略介绍。

\begin{verbatim}
#!/usr/bin/python2.5
#-*- coding: utf-8 -*-
from eurasia.web import config, mainloop, SimpleUpload, Response

page = '''\
<html>
<form action="/simpleupload" method="post" enctype="multipart/form-data">
<input type="hidden" name="a" value="表单变量A" />
<input type="file" name="z-file" />
<input type="submit" />
</form>
</html>
'''

def controller(request):
	if request.path != '/simpleupload':
		response = Response(request)
		response.write(page) # 输出前面定义的上传表单
		return response.close()

	fd = SimpleUpload(request)
	a = fd['a'] # <input type="hidden" name="a" value="..."/>

	filename = fd.filename
	size     = fd.size

	s1 = fd.read(10)      # 读取 10 个字符
	s2 = fd.readline(10)  # 读取一行 (最多读取 10 个字符)
	s3 = fd.read()[-20:]  # 读取所有内容, s3 保存了文件末尾的 20 个字符

	response = Response(conn) # 上传文件如果没有读取完毕, 可能会引起 IO 错误 
	response['Content-Type'] = 'text/plain; charset=utf-8'

	print >> response, '文件内容:'
	print >> response, s1 + s2, '...',

	print >> response, s3

	print >> resp, ''
	print >> resp, '变量 a:', a
	print >> resp, '文件名:', filename
	print >> resp, '文件大小:', size

	response.close()

config(controller=controller, port=8080)
mainloop()
\end{verbatim}

\note{SimpleUpload 要求表单变量名 (这里是 'a') 必须小于文件控件的名称 (这里是 'z-file')。因为对 SimpleUpload 组件来说, 文件组件的名字并不重要, 所以请尽量起成 "zzzzzzzz" 之类的名字, 以保证他的值是最大的。另外, SimpleUpload 组件只支持上传一个文件。当需要提交多个文件时, 你可以使用多个 SimpleUpload 组件, 这是 Eurasia 推荐使用多线程上传方式。}

\note{SimpleUpload 并不会真的把用户上传的文件建立在磁盘上, 也不会把用户的上传读取到内存中, 而是在调用 read() 时才会从 socket 中去获取。因此, 在没有读取完整个上传文件之前, 上传用户的浏览器将一直保持在 "正在上传" 的状态下。}

\note{HTTP 协议并不是每次都确切地知道用户上传文件的大小, SimpleUpload 能够猜测出用户上传内容的大小, 对于标准的浏览器和客户端这通常是准确的。在极少数情况下 SimpleUpload.size 和 SimpleUpload.read 取出的内容大小并不匹配, 这时应以 SimpleUpload.read 取出的实际内容大小为准, 一般来说, 你可以认为这是一种恶意行为。}

\section{使用 session}

你可以通过 Response['Set-Cookie'] 和 Request['Cookie'] 这两个接口直接操作 Cookie, 去实现一个自己喜欢的 Session 机制。

此外, Eurasia 也提供了 Session 支持。通过设置 Response.uid, 你可以为用户的浏览器指定一个永久的 Id (基于 Cookie), 并通过 Request.uid 取出。

\begin{tableii}{c|l}{}{ 属性 }{ 信息 }
\lineii{ request.uid          }{ 取得永久 Id }
\lineii{ response.uid = 永久Id }{ 设置永久 Id }
\end{tableii}

这里是一个简单的演示程序。创建并获取一个 uid。

\begin{verbatim}
#!/usr/bin/python2.5
#-*- coding: utf-8 -*-
import random, sha
from eurasia.web import config, mainloop, Response

def controller(request):
	response = Response(request)

	# 如果没有 uid 就为该浏览器用户分配一个
	if not request.uid:
		uid = sha.new('%s' %random.random()).hexdigest()
		response.uid = uid
		print >> response, 'uid %r created' %uid
		return response.close()

	# 有 uid 就在页面上输出
	print >> response, 'uid is %r' %request.uid
	response.close()

config(controller=controller, port=8080)
mainloop()
\end{verbatim}

基于永久 Id 可以做出各种 Session 实现。这里是一个内存 Session, 我们将用户提交的表单保存在 Session 中, 在下一次访问时取出。使用 dbm、bsddb、memcached 以及 MySQL 的原理相同。

\begin{verbatim}
#!/usr/bin/python2.5
#-*- coding: utf-8 -*-
import random, sha
from eurasia.web import config, mainloop, Form, Response

session = {}
newid = lambda: sha.new('%s' %random.random()).hexdigest()

def controller(request):
	response = Response(request)
	if not request.uid:          # 新建 session
		uid = newid()
		while uid in session:
			uid = newid()

		session[uid] = None
		response.uid = uid
	else:
		uid = request.uid

	data = session[uid]          # 前一次的表单数据
	session[uid] = Form(request) # 在 session 中保存本次的表单数据

	print >> response, data      # 返回前一次的表单数据
	response.close()

config(controller=controller, port=8080)
mainloop()
\end{verbatim}

\section{长连接}

Eurasia 主要是为长连接设计的, 在 Euraisa 中使用长连接就像呼吸一样自然。

可以发现 Eurasia 中的所有连接默认都是长连接的。正如前面所展示的那样, 所有的 Response 都需要显式地 close 掉, 否则连接将一直维持到用户断开。

除了常见的长连接页面输出, Eurasia 还提供了更为强大的基于长连接的 JavaScript 远程调用功能。

下面我们将进入 Eurasia 的长连接的世界。

\subsection{response 长连接模式}

通过 Response.begin() 调用, 我们可以让 response 对象工作在长连接模式下。下面是一个 response 长连接的例子, 在这个例子中, 每 2 秒服务器会向浏览器输出一次。

\begin{verbatim}
#!/usr/bin/python2.5
#-*- coding: utf-8 -*-
import eurasia
from eurasia.web import config, mainloop, Response

sleep = eurasia.modules['time'].sleep

def controller(request):
	response = Response(request)
	response['Content-Type'] = 'text/html; charset=utf-8'

	response.begin()
	print >> response, '<html>'

	for i in xrange(100):
		print >> response, '收到第 %d 条回复<br/>' % i
		sleep(2) # 间隔 2 秒发送一次

	print >> response, '</html>'
	response.end()

config(controller=controller, port=8080)
mainloop()
\end{verbatim}

一旦调用 response.begin() 以后, response.write() 会直接输出到浏览器, response.write() 函数会一直处于 "阻塞" 状态, 直至所有内容发送完毕。

如果在 response.begin() 和 response.write() 调用时与客户端的连接已经断开, 你会得到一个 Disconnected 异常。

\subsection{javascript rpc}

Eurasia 中的 Comet 对象是一种更为强大的长连接工具。服务器端可以在任何时候直接调用客户端的 JavaScript 函数 (JavaScript 远程调用)。

我们首先定义一个包含有 JavaScript 的 HTML 页面, 并与服务器建立一条长连接。

\begin{verbatim}
<html>
<head>
<script language="JavaScript">
function my_alert(stuff) { alert(stuff); };
</script>
</head>
<body>
<!-- 与 /remotecall 位置的服务器脚本建立长连接 -->
<iframe src="/remotecall" style="display: none;"></iframe>
</body
</html>
\end{verbatim}

下面是完整的处理脚本。

\begin{verbatim}
#!/usr/bin/python2.5
#-*- coding: utf-8 -*-
import eurasia
from eurasia.web import config, mainloop, Comet

page = '''\
<html>
<head>
<script language="JavaScript">
function my_alert(stuff) { alert(stuff); };
</script>
</head>
<body>
<!-- 与 /remotecall 位置的服务器脚本建立长连接 -->
<iframe src="/remotecall" style="display: none;"></iframe>
</body
</html>
'''

sleep = eurasia.modules['time'].sleep

def controller(request):
	if request.path != '/remotecall':
		response = Response(request)
		response.write(page) # 输出前面定义的上传表单
		return response.close()

	browser = Comet(request)
	browser.begin()              # JavaScript 远程调用开始
	for i in xrange(100):
		browser.my_alert(i)  # 调用客户端名为 "my_alert()" 的 JS 函数
		sleep(2)             # 间隔 2 秒发送一次


config(controller=controller, port=8080)
mainloop()
\end{verbatim}

我们同样可以通过 Comet['Header'] 和 Comet.uid 来设定响应的头部和持久 Id。

Comet.status 与 Comet.version 同样可以在 Comet 对象初始化时设置。

\begin{verbatim}
	comet = Comet(request, status=200, version='HTTP/1.1')
\end{verbatim}

下面是面向 Comet 对象的一份完整描述。

\begin{tableii}{c|l}{}{ 属性 }{ 信息 }
\lineii{ comet.uid                }{ 可以用于设置浏览器用户的持久 Id }
\lineii{ comet.status             }{ 响应状态, 常见的有 200 (正常响应, 默认), 404 (未找到) 等 }
\lineii{ comet.version            }{ 响应的 HTTP 版本, 'HTTP/1.1' (默认) 或 'HTTP/1.0' }
\lineii{ comet['Response-Header'] }{ HTTP 头, 注意 Key 是首字母大写的 }
\lineii{ getattr(comet, 'JavaScript-Function') }{ 取出客户端的 JavaScript, 用于远程调用 }
\end{tableii}

\section{深入 request 对象}

我们前面已经简单提到过 request 对象, 事实上 Eurasia 标准库包括 Form、Response、Comet、SimpleUpload 等工具都是基于 request 对象的。

对于熟悉 HTTP 的用户, 直接调用 request 接口要更简洁、更自由, 基于 request 能够得到更为强大的功能和更好的性能。

除了前面介绍的 request 属性之外, request 还带有读取 POST 报文和发送返回报文的功能, 这就是 request 的 read 和 write 接口。与 response 不同, request 能够直接向客户端发送原始请求报文, 包括返回 HTTP 状态以及 headers。

\begin{verbatim}
获取 POST 报文:
	data = request.read(size)
	data = request.readline(size)

输出 HTTP 响应报文:
	request.write(data)
\end{verbatim}

在下面的例子中, 服务器首先读取用户的 POST 报文, 然后发送响应报文。在这个例子里, 我们只用到了 request 对象而没有使用诸如 response 等额外的工具。

\begin{verbatim}
#!/usr/bin/python2.5
#-*- coding: utf-8 -*-
from eurasia.web import config, mainloop

def controller(request):
	print request.method, request.path, request.version # 输出请求状态
	for key, value in request.items():
		print key, ':', value                       # 输出请求头部

	if request.method == 'POST':
		print `request.read()`                      # 输出 POST 报文

	request.write('HTTP/1.1 200 OK\r\n')                # 输出响应状态
	request.write('Content-Type: text/html\r\n')        # 输出响应头部
	request.write('\r\n')
	request.write('<html>hello world!</html>')          # 输出页面
	request.close()

config(controller=controller, port=8080)
mainloop()
\end{verbatim}

上面我们已经基本涉及到了所有有用的 request 属性, 下面是一个汇总。

\begin{tableii}{c|l}{}{ 属性 }{ 信息 }
\lineii{ request.pid            }{ 当前 socket 连接的 fileno }
\lineii{ request.uid            }{ 如果已经设置了持久 Id, uid 属性可以取出该浏览器的持久 Id }
\lineii{ request.path           }{ 请求地址, 比如 '/index.html' }
\lineii{ request.method         }{ 请求方法, 取值是大写的 'GET' 或大写的 'POST' }
\lineii{ request.version        }{ 请求的 HTTP 版本, 'HTTP/1.0' 或 'HTTP/1.1' }
\lineii{ request['Http-Header'] }{ HTTP 请求头部, 比如 request['User-Agent'], 注意 Key 是首字母大写的 }
\lineii{ request.read(size)     }{ 读取指定大小的 POST 报文, 默认读取全部内容 }
\lineii{ request.readline(size) }{ 行读取 }
\lineii{ request.write(data)    }{ 写 HTTP 报文 }
\lineii{ request.close()        }{ 关闭与客户端的连接 }
\end{tableii}

\section{tcp 服务器设计}

除了 HTTP 服务器之外, Eurasia 还可以用于设计 TCP 服务器。

下面是使用 Eurasia 编写的一个简单的 echo 服务器。

\begin{verbatim}
#!/usr/bin/python2.5
#-*- coding: utf-8 -*-
from Eurasia.web import config, mainloop

# echo 服务器
def controller(request):
	while True:
		data = request.readline(1024) # 从 socket 读取
		if data == 'quit':
			request.close()
			break
		else:
			request.write(data)   # 输出到 socket

config(tcphandler=controller, port=8080)      # 与前面不同, 这里使用 tcphandler
mainloop()
\end{verbatim}

可以看到, 基本上使用 Eurasia 编写 TCP 服务器与编写 HTTP 服务器区别不大。

这里 TCP 服务器的 request 并不对 socket 中的内容作多余的处理, 你可以得到完整的 socket 输入, 并原样输出到 socket 客户端。

另一个区别在于 TCP 在 config 函数中是使用 tcphandler 指定的而不是原来的 controller。

在 Eurasia 中编写 TCP 服务器甚至要比编写一个 HTTP 服务器更简单, Eurasia 的 TCP 接口很少, 只涉及到 request 的输入输出。下面是在 TCP 服务器中可用的 request 属性。

\begin{tableii}{c|l}{}{ 属性 }{ 信息 }
\lineii{ request.pid            }{ 当前 socket 连接的 fileno }
\lineii{ request.read(size)     }{ 从 socket 中读取指定大小的内容, 默认读取全部内容 }
\lineii{ request.readline(size) }{ 行读取 }
\lineii{ request.write(data)    }{ 写 socket }
\lineii{ request.close()        }{ 关闭 socket 连接 }
\end{tableii}

这样, 基于 Eurasia 的 TCP 接口, 我们现在可以方便地设计出 FTP、SMTP 等基于 TCP 协议的服务器。

\note{TCP 服务器无法使用 Eurasia 中与 HTTP 相关的功能, 比如虚拟主机, 这在后面的章节中会讲到}

\subsection{一个简单的 HTTP 服务器}

Eurasia 的 HTTP 服务器正是基于 Eurasia 的 TCP 接口设计的, 可以发现这并不是一件十分困难的事情。

这是一个简单的示范, 使用浏览器连接到这个 TCP 服务器, 得到 hello world 的响应。

\begin{verbatim}
#!/usr/bin/python2.5
#-*- coding: utf-8 -*-
from eurasia.web import config, mainloop

def httpd(sockfile):
	# HTTP 协议第一行, 诸如 GET /PATH HTTP/1.1
	first = sockfile.readline(1024)

	# 第二行开始是 HTTP 协议头部, 诸如 NAME: VALUE 这种形式, 以空行结束
	headers = []
	line = sockfile.readline(1024).strip()
	while line: # 循环读取
		headers.append(line)
		line = sockfile.readline(1024).strip()

	body = sockfile.read() # 如果是 POST 请求, 则读取所有 Body 部分的内容

	# 返回响应关闭连接
	sockfile.write('HTTP/1.1 200 OK\r\n\r\nhello world!')
	sockfile.close()

config(tcphandler=httpd, port=8080)
\end{verbatim}

\section{使用 cherrypy 模拟器}

如果你熟悉 cherrypy, 对理解 Eurasia 的 cherrypy 模拟器会有一定帮助; 如果你不熟悉 cherrypy 那也没关系, Eurasia 只是借鉴了 cherrypy 的思想, 他们并不是一码事, 切勿对号入座。

下面是一个简单的示范。

\begin{verbatim}
#!/usr/bin/python2.5
#-*- coding: utf-8 -*-
from eurasia.web import config, mainloop, Response
from eurasia.cherrypy import expose, CherryEmulator

class MyRoot:
	@expose
	def __call__(self, request): # 默认页面, 相当于 cherrypy 中的 index 和 default
		response = Response(request)
		response.write('<h1>Default</h1>request.location is %r' %request.location)
		response.close()

	@expose
	def foo(self, request):      # 所有路径为 /foo/... 的请求 (包含 /foo) 都由该函数处理
		response = Response(request)
		response.write('<h1>Foo</h1>request.location is %r' %request.location)
		response.close()

	# 这个函数不使用 @expose 前置声明, 故该函数对用户不可见,
	# /bar 路径下的请求将由默认的 controller, 也就是前面的 __call__ 函数处理
	def bar(self, request):
		response = Response(request)
		response.write('<h1>Bar</h1>request.location is %r' %request.location)
		response.close()

controller = CherryEmulator(root=MyRoot()) # 使用 CherryEmulator 生成标准 controller

config(controller=controller, port=8080)
mainloop()
\end{verbatim}

CherryEmulator 实现了一种 url 到对象的映射, 根据请求的 url 找到对象中对应的方法进行处理。

在上面这个例子中, 存在如下映射关系。

\begin{itemize}

\item \strong{/... 映射至 MyRoot.__call__}

__call__ 方法定义了整个 MyRoot 对象的默认页面, 相当于目录下的 index.html 文件, 在 cherrypy 下叫做 index 或 default, 或者 zope 下的 index_html。如果 url 在 MyRoot 对象中没有找到对应的处理函数, 会默认使用对象的 __call__ 方法。

带有 __call__ 成员的对象事实上是一个函数对象, 可以想像, 使用 __call__ 作为默认页面, 是相当合适的。

\item \strong{/foo 映射至 MyRoot.foo}

所有发往包括 \file{/foo} 以及 \file{/foo/...} 的请求, 都由 MyRoot.foo 解析。

\end{itemize}

同样, 在这个例子中, 用到了 request 的 location 属性。这是 CherryEmulator 为 request 附加的新属性。对于 foo 方法来说, 如果用户请求的地址是 \file{/foo/a/b/c?a=1\&b=2} 那么 request.location 的取值是 \file{/a/b/c} (需要注意的是 \file{/foo} 和 \file{/foo/} 的取值是不同的, \file{/foo} 的 location 为空, 而 \file{/foo/} 为 \file{/})。

在 CherryEmulator 中, request 共有三种与请求 url 相关的属性。

\begin{tableii}{c|l}{}{ 属性 }{ 信息 }
\lineii{ request.path        }{ 完整的请求地址 }
\lineii{ request.location    }{ 未被解析的 url 部分, 不包括 query string }
\lineii{ request.querystring }{ url 中问号 "?" 后面部分的内容, 常用于传递以 GET 方式提交的表单信息 }
\end{tableii}

Eurasia 的 cherrypy 模块附带了大量的实用工具。下面, 我们将继续深入。

\note{cherrypy 模块是一个轻量级的应用层框架, 可以用来处理许多常见任务。一旦进入应用层, 功能的增长往往对应的是灵活性的快速下降, 如果你有信心基于 Eurasia 设计一个更顺手的应用层框架, 你并不需要基于 cherrypy 开始你的工作, 因为 cherrypy 模块已经引入了太多的限制。}

\subsection{多级路径}

这里将制造一个多级路径的处理程序。

程序中定义了一个多级对象, 发往 /a 和 /b 的请求分别由 A 和 B 对象处理, 发往 /a/a1 的请求则会由 A1 对象处理。

\begin{verbatim}
#!/usr/bin/python2.5
#-*- coding: utf-8 -*-
from eurasia.web import config, mainloop, Response
from eurasia.cherrypy import expose, CherryEmulator

class MyRoot:
	def __init__(self):
		self.a = A()                 # 将 A 对象挂接到 /a
		self.b = B()                 # 将 B 对象挂接到 /b

	@expose
	def __call__(self, request):         # /...
		response = Response(request)
		print >> response, '<h1>Root</h1>'
		response.close()

class A:
	def __init__(self):
		self.a1 = A1()               # 将 A1 对象挂接到 /a/a1 上

	@expose
	def __call__(self, request):         # 处理 /a 路径下的请求
		response = Response(request)
		print >> response, '<h1>A</h1>'
		response.close()

class A1:
	@expose
	def __call__(self, request):         # /a/a1
		response = Response(request)
		print >> response, '<h1>A1</h1>'
		response.close()

	@expose
	def a2(self, request):               # /a/a1/a2
		response = Response(request)
		print >> response, '<h1>A2</h1>'
		response.close()

class B:
	@expose
	def __call__(self, request):         # /b
		response = Response(request)
		print >> response, '<h1>B</h1>'
		response.close()

	@expose
	def b1(self, request):               # /b/b1
		response = Response(request)
		print >> response, '<h1>B1</h1>'
		response.close()

controller = CherryEmulator(root=MyRoot())

config(controller=controller, port=8080)
mainloop()
\end{verbatim}

\subsection{动态路径}

然而 url 往往是动态的。比如获取一个用户的基本信息, 我们常常会使用这样的 url: \file{/username/info}, 其中的 username 就是一个变量。

我们可以这样应对。

\begin{verbatim}
#!/usr/bin/python2.5
from eurasia.web import config, mainloop, Response
from eurasia.cherrypy import expose, CherryEmulator

class MyRoot(object):
	def __getattr__(self, name):
		return User(name)

class User:
	def __init__(self, name):
		self.name = name

	@expose
	def info(self, request):
		response = Response(request)
		print >> response, '<h1>my name is %s</h1>' %self.name
		response.close()

config(controller=CherryEmulator(root=MyRoot()), port=8080)
mainloop()
\end{verbatim}

在这个程序里, 如果你请求的 url 是 \file{/william/info} , 那么服务器会返回 my name is william 。

接下来如果把我们把站点的 root 对象指向对象数据库 (比如 ZODB 或者 Eurasia.shelve2) 那么会发生什么情况? 没错, 我们得到了 Zope!

\subsection{使用标准出错页面}

Eurasia CherryEmulator 为你准备了标准出错页面, 和 cherrypy 一样, 你只要抛出 NotFound 异常, CherryEmulator 就会为你输出标准的 404 页面。

\begin{verbatim}
	raise NotFound
\end{verbatim}

这里是异常与状态码的对应关系, 值得注意的是页面重定向 (Redirect) 也是以这种方式工作的。

\begin{tableii}{c|l}{}{ 状态码 }{ 异常 }
\lineii{ Forbidden           }{ 403 无权限 }
\lineii{ NotFound            }{ 404 未找到 }
\lineii{ InternalServerError }{ 500 服务器出错 }
\lineii{ Redirect            }{ 页面跳转 }
\end{tableii}

下面把各种效果都演示一下。

\begin{verbatim}
#!/usr/bin/python2.5
from eurasia.web import config, mainloop, Response
from eurasia.cherrypy import expose, CherryEmulator, \
	Forbidden, NotFound, InternalServerError, Redirect

class MyRoot(object):
	@expose
	def __call__(self, request):            # default page
		response = Response(request)
		response.write('<h1>Default</h1>')
		response.close()

	@expose
	def forbidden(self, request):           # 403 page
		raise Forbidden

	@expose
	def notfound(self, request):            # 404 page
		raise NotFound

	@expose
	def internalservererror(self, request): # 500 page
		raise InternalServerError

	@expose
	def google(self, request):              # www.google.com

		raise Redirect('http://www.google.com/')

config(controller=CherryEmulator(root=MyRoot()), port=8080)
mainloop()
\end{verbatim}

\note{只有在 CherryEmulator 中标准出错才是可用的, 在 CherryEmulator 之外, 并没有相应的自动处理机制, 那些都是普通的异常。}

\subsection{自定义标准出错页面}

CherryEmulator 对象 (controller) 本身即是一个 dict 对象, CherryEmulator[状态码] 中保存了默认的标准出错页面, 可以方便地覆盖为我们自己的标准出错页面。

在下面这个例子中, 当请求的地址无法找到, 或者访问了 \file{/notfound}, 访问者都将得到自定义的 404 页面 "mynotfound"。

\begin{verbatim}
#!/usr/bin/python2.5
#-*- coding: utf-8 -*-
from eurasia.web import config, mainloop, Response
from eurasia.cherrypy import expose, CherryEmulator, NotFound

page = '''\
<html>
<head>
	<title>404 Not Found</title>
</head>
<body>
	<h1>My Not Found</h1>
	<p>The requested URL %s was not found on this server.</p>
	<hr>
	<address>Eurasia Demo Server</address>
</body>
</html>'''

def mynotfound(request):        # 自定义 404 页面
	response = Response(request)
	response.write(page %request.path)
	response.close()

class MyRoot(object):
	@expose
	def foo(self, request):
		response = Response(request)
		response.write('<h1>Foo</h1>')
		response.close()

	@expose
	def notfound(self, request): # 404 page
		raise NotFound

controller = CherryEmulator(root=MyRoot())

controller[404] = mynotfound # 覆盖为自定义的 404 页面

config(controller=controller, port=8080)
mainloop()
\end{verbatim}

\subsection{标准文件服务器}

Eurasia 的 cherrypy 模块还提供了标准的静态文件输出服务。Directory 对象可以直接作为根 controller, 也可以挂接到 CherryEmulator 的二级路径下。

在这个例子里, url 根路径 \file{/} 被映射到磁盘静态目录 \file{/www} 下。例如用户访问了 \file{/test.txt} 这个 url, 那么, Directory 对象会去找到 \file{/www/test.txt} 这个文件返回给用户。

\note{如果没有找到指定的文件 Directory 对象会返回 404 Not Found 报文, 如果用户指定的是一个目录, Directory 对象会返回 403 Forbidden 报文。}

\note{列出磁盘目录高度依赖定制, 并考虑到安全性, 故没有默认提供}

\note{Directory 的磁盘 IO 主要是面向 Unix 的。}

\begin{verbatim}
#!/usr/bin/python2.5
#-*- coding: utf-8 -*-
from eurasia.web import config, mainloop
from eurasia.cherrypy import CherryEmulator, Directory

# 使用 Directory 来解析根路径, 把 url 映射到 /www 目录下
controller = CherryEmulator(root=Directory('/www'))
config(controller=controller, port=8080)

mainloop()
\end{verbatim}

Directory 对象也可以挂接到 CherryEmulator 对象的二级路径下。在这个例子里, Directory 被挂接到 \file{/static} 路径下。

在例子中 Directory 是 MyRoot 类的静态属性, 这也是最常见的设置方法。当然也可以将 Directory 设置为 MyRoot 的实例属性 (在 __init__ 初始化中进行设置)。

\begin{verbatim}
#!/usr/bin/python2.5
#-*- coding: utf-8 -*-
from eurasia.web import config, mainloop, Response
from eurasia.cherrypy import expose, CherryEmulator, Directory

class MyRoot(object):
	static = Directory('/www') # 挂接静态目录至 /static 下

	@expose
	def __call__(self, request):
		response = Response(request)
		response.write('<h1>Default</h1>')
		response.close()

controller = CherryEmulator(root=MyRoot())
config(controller=controller, port=8080)
mainloop()
\end{verbatim}

\subsection{定制标准文件服务器}

我们已经知道默认的 Directory 对象并没有将目录列出的功能, 而是直接返回 403 Forbidden 页面。我们可以通过定制, 实现列出目录的功能。

在这个例子中, 我们定制的是 Directory 类的 listdir 方法, 除了 self, listdir 还接受 request, dirname 两个参数。

\begin{verbatim}
#!/usr/bin/python2.5
#-*- coding: utf-8 -*-
import os, os.path
from eurasia.web import config, mainloop, Response
from eurasia.cherrypy import CherryEmulator, Directory

page = '''\
<html>
<head>
	<title>%s</title>
</head>
<body>
	<h1>%s</h1>
	<p>%s</p>
	<hr>
	<address>Eurasia Demo Server</address>
</body>
</html>'''

class MyDirectory(Directory):
	def listdir(self, request, dirname):
		# 首先尝试输出 index 索引页面
		for index in ['index.html', 'index.htm']:
			index = os.path.join(dirname, index)
			if os.path.exists(index):
				return self.fetch(request, index) # fatch 方法用于输出指定文件

		# 列出目录中的文件及子目录
		l1, l2 = ['..'], []
		for i in os.listdir(dirname):
			filename = os.path.join(dirname, i)
			if os.path.isdir(filename):
				l1.append(i)
			else:
				l2.append(i)

		l1.sort(); l2.sort()
		fldrlst = ['<a href="%s/">%s/</a>' %(i, i) for i in l1]
		filelst = ['<a href="%s">%s</a>'   %(i, i) for i in l2]

		# 输出页面
		response = Response(request)
		response['Content-Type'] = 'text/html'
		print >> response, page %(request.location, request.location,
			'<br />\n'.join(fldrlst + filelst))

		response.close()

config(controller=CherryEmulator(MyDirectory('/www')),
	port = 8080)

mainloop()
\end{verbatim}

Directory 类的 fetch 方法, 用于返回用户请求的单个文件。我们可以通过重载 Directory 的 fetch 方法来实现一些特定的功能, 比如隐藏 \file{.htaccess} 和 \file{.htpasswd} 文件。

\begin{verbatim}
#!/usr/bin/python2.5
#-*- coding: utf-8 -*-
import os.path
from eurasia.web import config, mainloop
from eurasia.cherrypy import CherryEmulator, Directory, NotFound

class MyDirectory(Directory):
	def fetch(self, request, filename):
		# 隐藏以 .ht 开头的文件
		if os.path.split(filename)[1][:3] == '.ht':
			raise NotFound

		# 调用原 fetch 方法
		return Directory.fetch(self, request, filename)

config(controller=CherryEmulator(MyDirectory('/www')),
	port = 8080)

mainloop()
\end{verbatim}

\subsection{第一个 cgi 服务器}

上面, 我们通过定制 Directory 对象的 fetch 方法实现了 \file{.ht} 文件的隐藏。现在, 我们进一步, 通过修改 fetch 制作出我们自己高度定制的 cgi 服务器。

在这个例子中, 实现了一个 cgi 服务器 (出于演示目的, 只是简单地 dup 了 socket)。

\begin{verbatim}
#!/usr/bin/python2.5
#-*- coding: utf-8 -*-
import sys, os, os.path, copy
from urllib import unquote_plus
from eurasia.web import config, mainloop, Form, Response
from eurasia.cherrypy import CherryEmulator, Directory, NotFound, InternalServerError

environ = copy.copy(os.environ)

class CgiDirectory(Directory):
	def fetch(self, request, filename):
		if not os.path.exists(filename):
			raise NotFound

		try:
			test = executable(filename)
		except OSError:
			raise InternalServerError

		if not test:
			return Directory.fetch(self, request, filename)

		if os.fork() != 0:
			return

		environ.update(Form(request.querystring)) # 将表单设置到环境变量中
		environ['LOCATION']       = request.location
		environ['QUERY_STRING']   = request.querystring
		environ['REQUEST_METHOD'] = request.method
		try:
			environ['HTTP_COOKIE'] = request['Cookie']
		except KeyError:
			pass
		try:
			environ['CONTENT_LENGTH'] = request['Content-Length']
		except KeyError:
			pass

		# request.pid 即请求连接的 fileno
		# 重定向标准 IO 至 socket
		os.dup2(request.pid,  sys.stdin.fileno())
		os.dup2(request.pid, sys.stdout.fileno())

		# 运行 cgi 程序
		os.execvpe(filename, (filename, ), environ)

def executable(filename):
	st = os.stat(filename)
	return st.st_mode & 0111 != 0

def Form(query): # 解析通过 GET 方式提交的表单信息
	dct = {}
	for item in query.split('&'):
		try:
			key, value = item.split('=', 1)
		except ValueError:
			continue
		else:
			dct[key] = unquote_plus(value)
	return dct

config(controller=CherryEmulator(CgiDirectory('/www')),
	port = 8080)

mainloop()
\end{verbatim}

下面, 我们在 \file{/www} 目录下编写一个名为 \file{test.py} 的 cgi 程序。

\begin{verbatim}
#!/usr/bin/env python
import sys, cgi, cgitb; cgitb.enable()

form = cgi.FieldStorage()
response = '%s %s!' %(form['a'].value, form['b'].value)

print 'HTTP/1.1 200 OK'
print 'Content-Type: test/plain'
print
print response
\end{verbatim}

使用命令 "chmod +x test.py" 将程序设置为可执行。启动 cgi 服务器以后, 我们可以在浏览器中输入地址 \file{/test.py} (结果是 Python 标准 cgi 库的出错页面) 和 \file{/test.py?a=hello\&b=world} (结果是 "hello world!") 分别进行测试。

\note{原理演示, 切勿实用!}

\section{模拟 django}

因为可以从 request.path 得到完整的 url, 所以可以较为方便地模拟出 django 的 url 映射。

下面是 Eurasia 对于 django 的 url 处理模式的模拟。

\begin{verbatim}
#!/usr/bin/python2.5
#-*- coding: utf-8 -*-
from eurasia.web import config, mainloop, Response

def hello(resp, *args, **kw):
	print >> resp, 'hello:', args, kw
	resp.close()
def world(resp, *args, **kw):
	print >> resp, 'world:', args, kw
	resp.close()

urlpatterns = (
	(r'^/hello/[^/]+/(?P<foo>.*)', hello),
	(r'^/world/[^/]+/(?P<bar>.*)', world)  )

def controller(client):
	path = client.path
	for pattern, func in urlpatterns:
		match = pattern.search(path)
		if match:
			func(Response(client), *(match.groups()),
				**(match.groupdict()) )
			return

	print >> client, '404'
	client.shutdown()

urlpatterns = tuple(
	[(compile(pattern), func) for pattern, func in urlpatterns])

config(controller = controller, port = 8080)

mainloop()
\end{verbatim}

\note{这里模拟了 django 的工作方式, 但是省略了对于正则表达式的排序操作, 采用手工排序。}

\note{在 CherryEmulator 中, 可以通过 request.location 和 request.querystring 两个实用属性混用 cherrypy 和 django 两种工作模式。在这种情况下, 可以省略相当一部分路径解析的工作, 我们可以很方便地实现正则表达式对于路径的一次扫描。正则表达式对于路径实现单次扫描对于性能的提高有一定意义。}

\section{使用标准模板}

Eurasia 的 template 模块是 Mako 模板的一个简化版, 带有大部分 Mako 模板的功能, 和相同的标签语法。同时 Eurasia 标准模板也是编译型的模板。

template 包涵以下标签语法。

\begin{itemize}

\item \strong{表达式替换}

\begin{verbatim}
this is x: ${x}
\end{verbatim}

进一步, 取值表达式可以嵌入 Python 代码, 并替换为 Python 表达式的运算结果。

\begin{verbatim}
${int(a) + int(b)}
\end{verbatim}

\item \strong{控制结构}

我们可以在模板中使用条件表达式及叠代循环表达式。

这里是条件表达式。

\begin{verbatim}
%if x==1:
	x is ${x}
%elif x==2:
	x is ${x}
%else:
	x is ${x}
%endif
\end{verbatim}

这里是循环。

\begin{verbatim}
%for a in ['one', 'two', 'three', 'four', 'five']:
	%if a[0] == 't':
		its two or three
	%elif a[0] == 'f':
		four/five
	%else:
		one
	%endif
%endfor
\end{verbatim}

\item \strong{Python 代码}

可以在模板中运行 Python 代码。

\begin{verbatim}
<% # 这里是 Python 代码
a = 1
b = 2
%>

测试一下 a + b : ${a + b}
\end{verbatim}

\note{第一行 Python 代码的缩进会被忽略, 所以请避免使用 def、class、if、for 等依赖缩进的 Python 代码。函数中的 Python 代码块在当前函数中可见, 函数外的代码块在当前模板中全局可见。}

\item \strong{函数标签 <\%def name="..." \%>}

函数是模板中最基本的调用单位, 类似于 Python 中的函数。

\begin{verbatim}
<%def name="myfunc(x)">
    this is myfunc, x is ${x}
</%def>

调用: ${myfunc(7)}
\end{verbatim}

\item \strong{<\%call expr="..."\%> 标签}

call 标签用于调用 <\%def\%> 标签,可传递额外的内嵌内容。稍后介绍。

\end{itemize}

\subsection{使用模板}

为了演示 Eurasia 标准模板的用法, 这里将建立一个模板的应用范例。

在这个例子里, 我们首先将编写一个字符串模板, 带有 test1 和 test2 两个函数。然后使用 template 模块中的 Template 函数将字符串模板编译成可执行的 Python 模块, 并对模板中的 test1、test2 函数进行调用。

\begin{verbatim}
#!/usr/bin/python2.5
#-*- coding: utf-8 -*-
from eurasia.template import Template

s = '''\                  # 字符串模板
<%def name="test1(a)">
	test1, a is ${a}
</%def>

<%def name="test2(b)">
	test2, b is ${b}
</%def>'''

tmpl = Template(s)        # 将字符串模板编译成可执行的 Python 模块

print tmpl.test1('hello') # 调用模板中的 "test1" 函数
print tmpl.test2('world') # 调用模板中的 "test2" 函数
\end{verbatim}

结果是

\begin{verbatim}
test1, a is hello
test2, b is world
\end{verbatim}

<\%def> 标签定义了模板的调用接口, 在这个例子中, 我们定义了 test1 和 test2 两个调用接口。

\note{我们已经确认 CPython 中存在的一个 Bug 将导致 Template(s).test1('hello') 这种写法报错。因此, 在 CPython 中需要分开来写成 tmpl = Template(s); tmpl.test1('hello')。也就是上面这个例子中的写法。}

\subsection{一个更为复杂的例子}

本例中, 将用到大量模板常用语法。包括有条件判断, 循环等。

\begin{verbatim}
#!/usr/bin/python2.5
#-*- coding: utf-8 -*-
from eurasia.template import Template

s = '''\
<%def name="main(lst)">
--------------------------------------------
<% # Python 代码块示范
class Foo:
	def test(self):
		return 'from foo.test()'

foo = Foo()

# 在 Python 代码块中可以使用 write 函数进行内容输出
write(foo.test())

%>

${Foo().test()}

--------------------------------------------
# 循环判断示范
%for i in lst:
	%if i == 1:
		${1+2}
	%elif i == 2:
		${'hello' + 'world'}
	%else:
		${i}
	%endif
%endfor

--------------------------------------------
# 嵌入式函数示范, 只在 main 函数中可见
<%def name="bar()">
	this is bar
</%def>
${bar()}

--------------------------------------------
</%def>'''

tmpl = Template(s)
print tmpl.main([1, 2, 3, 4, 5])
\end{verbatim}

\subsection{使用 <\%call> 标签定义宏}

<\%call> 标签提供了一种较为高级的功能, 也就是模板宏。

\begin{verbatim}
<%def name="macros(a, b)">
${caller.slot1(a)}
${caller.slot2(b)}
</%def>

----------------------------

<%def name="main()">
<%call expr="macros(1, 2)">
  <%def name="slot1(a)">
    slot1, ${a}
  </%def>

  <%def name="slot2(b)">
    slot1, ${b}
  </%def>
</%call>
</%def>
\end{verbatim}

这里首先使用 <\%def> 定义了一个名为 macros 的模板宏。

然后在 main 函数中使用 <\%call> 调用 macros, 其中定义了 slot1 和 slot2, 在 macros 中可以通过 caller 取出。

该模板的入口函数是 main()。调用结果是

\begin{verbatim}
slot1, 1
slot2, 2
\end{verbatim}

\note{本例中在同一个模板中同时包涵了模板宏和宏调用, 通常情况下, 我们会将模板宏保存入外部文件。}

\note{call 只能在函数中 (<\%def>) 调用。如果把 call 放置在模板顶层, 模板在编译时会忽略 call 调用。}

\subsection{缓存编译结果}

Eurasia 标准模板是一种编译型模板, 字符串模板经过 Template 函数可以编译成可执行的 Python 模块。

我们也可以仅仅把字符串模板转换成 Python 源代码, 在文件中保存起来, 这样可以在下次使用时直接 import 进来, 省去了编译过程, 尽管这花不了多少时间。

这里我们将用到 template 的 compile 工具。

\begin{verbatim}
#!/usr/bin/python2.5
#-*- coding: utf-8 -*-
from eurasia.template import compile

s = '''\
<%def name="main()">
	hello world!
</%def>

# 得到 Python 源码
code = compile(s)

# 保存为 Python 模块文件
fd = open('cache.py', w)
fd.write(code)
fd.close()

# 以模块方式导入模板
import cache
print cache.main()
\end{verbatim}

\subsection{<\%namespace> 标签}

Euraisa 标准模板中没有 Mako 中 <\%namespace> 标签的等价物, 这个功能相当于 Python 中的 import。<\%namespace> 需要一系列关于导入路径的配置, 这比较复杂, 因此 Eurasia 换了一种方式。

Template 函数第二个参数可以设定模板中可见的环境。你可以在这里预先导入一些模板中用得到的东西, 比如宏。

这个例子在讲解 <\%call> 时出现过, 这里把宏和调用部分拆离开来。

\begin{verbatim}
#!/usr/bin/python2.5
#-*- coding: utf-8 -*-
from eurasia.template import Template
s1 = '''\ # 定义模板宏
<%def name="macros(a, b)">
${caller.slot1(a)}
${caller.slot2(b)}
</%def>'''

s2 = '''\ # 调用模板宏
<%def name="main()">
<%call expr="macros(1, 2)">
  <%def name="slot1(a)">
    slot1, ${a}
  </%def>

  <%def name="slot2(b)">
    slot1, ${b}
  </%def>
</%call>
</%def>'''

tmpl1 = Template(s1)

# 使用 env 参数指定模板环境
tmpl2 = Template(s2, env={'macros': tmpl1.macros})

tmpl2.main()
\end{verbatim}

\section{使用对象数据库}

Eurasia 带有一种极其轻便的嵌入式对象数据库 shelve2。如果用户具有 ZODB 或者 durus 等对象数据库的使用经验, 会有助于理解 shelve2 的使用方法和工作方式, 但是这不是必须的, 因为 shelve2.py 要简单得多。

\begin{verbatim}
#!/usr/bin/python2.5
#-*- coding: utf-8 -*-
#from eurasia.shelve2 import open        # 即时 IO
from eurasia.shelve2 import lazy as open # 惰性 IO

d = open('test.fs', 'c') # 第二个参数是数据库文件的打开方式,
                         # 'c' 是指以读写方式打开, 如果文件不存在则自动创建
                         # 默认使用 'r' 以只读方式打开

d['foo'] = {'a': 1, 'b': 2, 'c': 3}

#d.sync() # 手动保存
d.close() # 数据库关闭时会自动保存

# 再次打开
d2 = open('test.fs')
print d2['foo']
\end{verbatim}

\subsection{对象数据库}

在对象数据库中, Python 对象可以直接保存在数据库中, 而无须像关系数据库那样经过 "对象关系映射"。

下面, 我们将定义一个 MyObject 对象并存入 shelve2 数据库。

\begin{verbatim}
#!/usr/bin/python2.5
#-*- coding: utf-8 -*-
from shelve2 import lazy as open

class MyObject:
        def __init__(self):
                self.a = None

        def print_a(self):
                print self.a

d = open('test.fs', 'c')
d['myobj'] = MyObject()
d['myobj'].a = 'hello world!'
d.close()

# 注意, 在打开数据库时, 对象定义 class MyObject 必须存在
# 否则会得到 AttributeError 的异常
d2 = open('test.fs')
d2['myobj'].print_a()
\end{verbatim}

\subsection{使用 Persistent 创建持久对象结点}

\begin{verbatim}
#!/usr/bin/python2.5
#-*- coding: utf-8 -*-
from shelve2 import lazy as open
from shelve2 import Persistent

d = open('test.fs', 'c')
obj = d.new(Persistent)() # 创建持久对象结点,
                          # 显式指定持久对象归属于数据库 'd'

obj.a = 'hello'
obj.b = 'world'
d['obj'] = obj
d.close()

d2 = open('test.fs')
obj = d2['obj']
print '%s %s!' %(obj.a, obj.b)
\end{verbatim}

在对象数据库中, Persistent 不同于一般的 Object。他们的区别在于在 Persistent 对象和普通 Object 同时作为父对象的属性时, 当父对象被展开、取出对象数据库时, 作为父对象属性的常规 Object 也会被自动展开读取到内存中; 而 Persistent 对象在父对象展开时仅以指针形式存在, 不会被展开到内存中。所以 Persistent 对象更节省内存并节约 IO, 这使得 Persistent 对象非常适合作为数据节点。

你可能注意到于 ZODB 或者 Durus 数据库不同, shelve2.py 需要显式指定持久对象的数据库归属。这正是 shelve2.py 被设计出来的重要原因之一。ZODB 和 Durus 会为每一个线程隐含指定一个互相独立的连接 (Connection), 但是对于 Stackless Python 的 tasklet 线程来说, 这或许不大方便 (Eurasia3 正是基于 Stackless Python 的), 因为 tasklet 和常规意义上的线程还是有很大的不同。所以 shelve2.py 认为有必要把连接的选择权交还给用户。 

\subsection{定制 Persistent 对象}

自定义 Persistent 对象需从 Persistent 类继承。

\begin{verbatim}
#!/usr/bin/python2.5
#-*- coding: utf-8 -*-
from shelve2 import lazy as open
from shelve2 import Persistent

class Foo(Persistent):
	def __init__(self, a, b):
		self.a = a
		self.b = b

	def print_ab(self):
		print '%s %s!' %(self.a, self.b)

d = open('test.fs', 'c')
d['foo'] = d.new(Foo)('hello', 'world')
d.close()

d2 = open('test.fs')
d2['foo'].print_ab()
\end{verbatim}

\subsection{使用 BTree 创建大容量数据结点}

BTree 对象可以用以保存数以千万计的对象, 同时被保存的这些对象是按照 Key 排序的, 你可以以任何方式定位、遍历这些内容。

\begin{verbatim}
#!/usr/bin/python2.5
#-*- coding: utf-8 -*-
from shelve2 import lazy as open
from shelve2 import BTree

d = open('test.fs', 'c')
tr = d.new(BTree)()
for i in xrange(1000):
	tr[i] = `i`

d['tr'] = tr
d.close()

d2 = open('test.fs')
print d2['tr']
\end{verbatim}

BTree 除了支持所有类似于 dict 的方法外, 还支持各种遍历方法:

\begin{tableii}{c|l}{}{ 说明 }{ 属性 }
\lineii{ 正向遍历                 }{ BTree.iter() }
\lineii{ 反向遍历                 }{ BTree.reversed() }
\lineii{ 正向遍历返回键、值 }{ BTree.iteritems() }
\lineii{ 反向遍历返回键、值 }{ BTree.items_backward() }
\lineii{ 正向遍历返回键、值 }{ BTree.items_from(key, closed=True) }
\lineii{ 反向遍历返回键、值 }{ BTree.items_backward_from(key, closed=True) }
\lineii{ 遍历区域返回键、值 }{ BTree.items_range(start, end, closed_start=True, closed_end=False) }
\end{tableii}

BTree 对象可以以和 Persistent 对象类似的方法派生。

shelve2 对象数据库的根对象就是一个 BTree 对象。 

\subsection{Persistent 和 BTree}

在 Zope/Plone 中, 不仅文件是 Persistent 对象, 连文件夹也是 Persistent 对象, 此时一个文件就是 Persistent 对象的一个属性。只有在文件夹过大的时候我们才会使用从 BTree 派生出来的 PloneLargeFolder。所以从本质上说, Persistent 和 BTree 都是容器对象, 他们的区别在于 Persistent 对象的内容在数据库里是连续存储, 而在内存中则变成 Hash Table, 在展开 Persistent 对象时其所有的属性也会随之一次性全部展开; 而 BTree 对象无论在数据库中还是内存中都是以平衡树的方式存储的, 读取 BTree 对象时 BTree 对象中的内容不会一次完全展开, 而是按需要分批展开。

所以在需要频繁读取容器内容时、并且容器中放置的内容并不多的情况下 Persistent 会快于 BTree。而当保存的内容很多时 BTree 要更快。但是我们往往很难知道某个容器将来到底会扩充到何种规模, 所以可以在任何时候都使用 BTree 作为容器, 在性能上这其实并无大碍。

\section{配置文件支持}

Eurasia 的 pyetc 模块为配置文件的读取提供了支持。Eurasia 认为使用 Python 语法来编写配置文件是个好主意。

pyetc 中的 load(filename, env={}) 函数能读取指定的 python 源码文件 (文件后缀并不需要一定是 ".py"), 并设定配置文件中的可见环境 env。load 函数能将指定源码文件转换成 Python 模块并返回。

我们首先编写一个名为 \file{httpd.conf} 的配置文件。

\begin{verbatim}
Server(controller='Products.default.controller'
	port=8080)

user = 'nobody'
\end{verbatim}

下面是对于这个配置文件的解析

\begin{verbatim}
from eurasia import pyetc
config = {}
def Server(**args):
	config.update(args)

mod = pyetc.load('httpd.conf', env={Server:Server})
print 'Server:', config
print 'user', mod.user
\end{verbatim}

\section{虚拟主机}

此前我们已经接触过 Eurasia 的 config 函数, 可以用于定义 HTTP 和 TCP 服务器。除此之外, 对于 HTTP 服务器, config 还可以进行更为深入细致的配置。

\subsection{多 IP, 多端口}

此前我们一直是将服务器绑定到本机的某个端口发布的。现在, 如果我们要将服务器发布到多个 IP 和端口上该怎么做呢?

在这个例子里, 我们将一个服务器绑定到 192.168.0.1 地址的 80 端口和 192.168.0.2 地址的 8080 端口。

\begin{verbatim}
#!/usr/bin/python2.5
#-*- coding: utf-8 -*-
from eurasia.web import config, mainloop
def controller(request):
	request.close()

# 分别绑定到 192.168.0.1 的 80 端口, 和 192.168.0.2 的 8080 端口
config(bind='192.168.0.1:80, 192.168.0.2:8080',
	controller=controller)
\end{verbatim}

我们可以同时配置多个服务器。

\begin{verbatim}
#!/usr/bin/python2.5
#-*- coding: utf-8 -*-
from eurasia.web import config, mainloop
def controller1(request): request.close()
def controller2(request): request.close()

# 分别绑定到 192.168.0.1 的 80 端口, 和 192.168.0.2 的 8080 端口
config(bind='192.168.0.1:80, 192.168.0.2:8080',
	controller=controller1)

# 分别绑定到 192.168.0.3 的 80 端口, 和 192.168.0.4 的 8080 端口
config(bind='192.168.0.3:80, 192.168.0.4:8080',
	controller=controller2)

\end{verbatim}

\note{TCP 服务器和 HTTP 服务器都可以使用多 IP 多端口绑定}

\note{不可以将 TCP 服务器和 HTTP 服务器绑定到同一个地址的同一个端口}

\note{只有在使用了虚拟主机的情况下多个 HTTP 服务器才可以重复绑定到同一个地址的同一个端口}

\subsection{虚拟主机}

config 支持虚拟主机的设定。比想像得还要简单。

这里我们将绑定到 www.example.com 和 eurasia.example.com 这两个主机。

\begin{verbatim}
#!/usr/bin/python2.5
#-*- coding: utf-8 -*-
from eurasia.web import config, mainloop
def controller(request): request.close()

# 分别绑定到 192.168.0.1 的 80 端口, 和 192.168.0.2 的 8080 端口
# 仅处理指定主机为 *.www.example.com 和 *.euraisa.example.com 的请求
config(
	bind='192.168.0.1:80, 192.168.0.2:8080',
	host='www.example.com, eurasia.example.com', # 指定虚拟主机

	controller=controller)
\end{verbatim}

\note{虚拟主机功能不支持 TCP 服务器}

\section{实用工具}

这里将介绍一些 Eurasia 中自带的实用工具。

\subsection{sleep}

有时我们需要定时唤醒功能, 但 Eurasia 轻便线程 (micro thread) 中是不允许阻塞和 IO 操作的 (CPU 和内存操作是允许的), 因此 Eurasia 提供了一个非阻塞的 sleep 实现, 当轻便线程中执行 sleep 函数时, CPU 会跳出并执行其他线程, 直到 sleep 指定的时间结束以后再回到该线程继续执行。

Eurasia 把 Python 标准库中阻塞实现的修改版都放在 eurasia.modules 包中。因为 sleep 在 Python 标准库的 time 模块下, 因此我们可以这样取出 sleep 函数

\begin{verbatim}
import eurasia
sleep = eurasia.modules['time'].sleep
\end{verbatim}

\note{使用 sleep 会启动时间管理器, Eurasia 整体性能会略有下降}

\subsection{urlopen}

Eurasia 提供了标准库中 urlopen 的非阻版, 同时增加了设定 HTTP Headers 的功能。

你可以这样取出 urllib。

\begin{verbatim}
import eurasia
urlopen = eurasia.modules['urllib'].urlopen
\end{verbatim}

urlopen 的参数是:

\begin{verbatim}
response = urlopen(url, data='', headers={}, **args)
\end{verbatim}

其中, data 是发送的 POST 报文, 如果不指定 data 那么 urlopen 将发送 GET 请求。

headers 可以指定 HTTP Headers, 也可以以 args 的方式指定 Headers。这里是一个例子:

\begin{verbatim}
fd = urlopen('http://www.example.com/', data='a=1&b=2',
	{'Accept-Charset': 'iso-8859-5, unicode-1-1;q=0.8'},
	Accept_Encoding= 'gzip;q=1.0, identity; q=0.5, *;q=0' # args 方式, 使用 "_" 代替 "-"
	)

# 输出响应状态
print fd.version, fd.status, fd.message

# 输出响应的头部
for key, value in fd.items():
	print key, ':', value

print fd.readline() # 读取一行
print fd.read()     # 读取全部
\end{verbatim}

可以发现, urlopen 返回的文件句柄 fd 的使用方式和 Eurasia HTTP 服务器的 request/response 都是一致的。

\note{在 Eurasia 3.0.0 版本中, urlopen 仅支持 HTTP 协议}

\note{因为可以处理 Headers, 所以可以直接通过 Set-Cookie 和 Cookie 头来处理 cookie}

\note{urlopen 会原样返回远程 HTTP 服务器的响应报文, 而不会自动执行 gzip 等解码 (如果远程服务器确实使用了 gzip 等编码)}

\subsection{磁盘文件}

在 Unix 下文件操作同样可以使用异步 IO。这时你需要用 eurasia.web.File 替换系统的 open 函数来打开文件, 参数是一样的。 

\begin{verbatim}
from eurasia.web import File as open
\end{verbatim}

\note{Eurasia 异步文件只能在 Unix 中使用, Windows 用户请参考 google code 项目 stackless example 中的 \file{stacklessfile.py}}

\section{mkeurinstance.py}

Eurasia 中的 \file{mkeurinstance.py} 脚本可以帮你建立一个简单的 Eurasia 站点。\file{mkeurinstance.py} 脚本的使用很简单, 你只需要在控制台下输入

\begin{verbatim}
/PATH/TO/python2.5 /PATH/TO/mkeurinstance.py
\end{verbatim}

\file{mkeurinstance.py} 脚本会询问你在哪里建立 Eurasia 站点, 并在目标路径建立 Eurasia 站点。

自动生成的 Eurasia 站点包含 \file{bin}、\file{etc}、\file{lib}、\file{var}、\file{tmp}、\file{Products} 这几个目录。

\note{你用来执行 mkeurinstance.py 脚本的 python, 就是生成站点所用的 python。如果你机器上有多个 python, 执行 mkeurinstance.py 脚本的时候就是选择用哪个 python 来运行 Eurasia 站点的时机。}

\subsection{配置服务器}

Eurasia 站点的 etc 目录保存了站点的配置文件, 主要是 \file{eurasiad.conf}。下面简介一下配置文件格式。

\begin{itemize}

\item \strong{Daemon 块}

\file{eurasiad.conf} 中的 Daemon 块指定了 Eurasia 服务器后台进程, 包括 address (pid 文件的位置) 和 program (服务器主程序)。

\begin{verbatim}
Daemon(
	address = var('eurasiad.pid'),
	program = bin('runeurasia' )
)
\end{verbatim}

\note{配置文件中可以使用 bin、etc、lib、var、tmp 等预设函数, 函数返回分别返回文件在站点中的相关位置。}

\item \strong{Server 块}

\file{eurasiad.conf} 中的 Server 块相当于 config 函数, 可以有多个。

\begin{verbatim}
Server(
	controller = Products.default.controller,
	bind = '0.0.0.0:80, 0.0.0.0:8080',
	host = 'www.example.com, eurasia.example.com'
)
\end{verbatim}

Server 块和 config 函数不同的地方在于其 controller (或者 tcphandler) 设置。上面用到了 Products.default.controller, 意为 Products 目录下的 default 包中的 controller 函数。你也可以用字符串的方式来指定

\begin{verbatim}
controller = "Products.default.controller"
\end{verbatim}

\item \strong{multicore 参数}

你可以通过 multicore 指定服务器进程数。默认为 True, Eurasia 会自动检测服务器 CPU 数量。如果设置为 False, 则使用单进程。也可以直接指定一个进程数。

\begin{verbatim}
multicore = 2 # CPU 或者 CPU 核心数量, 默认自动检测当前 CPU 数, 设置 False 为单核
\end{verbatim}

\note{需要注意的是一旦使用了多进程, 各个微进程之间将是进程间通信, 你将不能享受到无锁单线程的便利, 这是一个及其巨大的损失。因此在许多场合你需要把 multicore 设置成 False。我们只能期待 PyPy 将来能够解决这个问题。}

\item \strong{user 参数}

你可以设置服务器以哪个用户的身份来执行。默认是当前用户。

\begin{verbatim}
user = 'nobody'
\end{verbatim}

\note{如果设置了 user, 为了切换用户, 你需要以 root 身份来启动服务器。}

\item \strong{verbose 参数}

调试模式默认是关闭的, 你可以在配置文件中这样打开调试模式。

\begin{verbatim}
verbose = True
\end{verbatim}

\end{itemize}

\subsection{启动服务器}

\begin{itemize}

\item \strong{runeurasia}

\file{bin} 目录下的 \file{runeurasia} 脚本是服务器的主程序。你可以执行该脚本将服务器启动在控制台中。如果在配置文件中设定了 verbose = True, 那么可以看到错误输出。

\begin{verbatim}
$ cd /PATH/TO/bin
$ ./runeurasia
\end{verbatim}

\note{我们通常在调试时使用 runeurasia}

\item \strong{eurasiad}

\file{eurasiad} 脚本可以以后台进程的方式启动服务器 (精灵进程)。

\begin{verbatim}
$ ./eurasiad start
\end{verbatim}

你可以使用 eurasiad stop 停止服务器, 并通过 eurasiad status 来检查服务器状态。

eurasiad start 在启动服务器的同时还会启动 daemon 管理进程, 你可以通过 eurasia tiny [procname] 的形式仅启动服务器而不使用 daemon 管理器。这时你只能通过 killall procname 的形式来停止服务器。

\begin{verbatim}
$ ./eurasiad tiny hello
$ killall hello
\end{verbatim}

\note{只有在 Unix 下才可以使用 eurasiad 脚本, Windows 下请使用 pywin32 将服务器注册为服务, 或者使用 taskbar 组件将服务器运行在托盘区}

\end{itemize}

\subsection{Products 目录}

我们前面已经看到在配置文件中有专门指向 Products 的连接对象。我们编写的站点和第三方扩展都放置在 Products 下。

对于 Eurasia 站点中的程序来说, Products 是一个可见的包, 我们可以在程序中这样得到 Products 下的内容。

\begin{verbatim}
import Products
from Products import ...
\end{verbatim}

除了 Products 之外, 站点的 lib 目录也已经被加入了 Python 的库搜索路径, 你可以把需要用到的库放置在 Products 和 lib 目录下, 而无须把这些东西安装在 Python 上。

\end{document}
